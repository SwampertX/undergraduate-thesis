% Load the kaobook class
\documentclass[
	fontsize=10pt, % Base font size
	twoside=true, % Use different layouts for even and odd pages (in particular, if twoside=true, the margin column will be always on the outside)
	%open=any, % If twoside=true, uncomment this to force new chapters to start on any page, not only on right (odd) pages
	secnumdepth=2, % How deep to number headings. Defaults to 1 (sections)
% ]{kaohandt}
]{kaobook}

% Choose the language
\usepackage[english]{babel} % Load characters and hyphenation
\usepackage[english=british]{csquotes}	% English quotes

% Load packages for testing
% \usepackage{blindtext}
%\usepackage{showframe} % Uncomment to show boxes around the text area, margin, header and footer
%\usepackage{showlabels} % Uncomment to output the content of \label commands to the document where they are used

% Load the bibliography package
\usepackage{kaobiblio}
\addbibresource{main.bib} % Bibliography file

% Use Meven's PhD paper's style
\usepackage{styles/layout}

% Load mathematical packages for theorems and related environments
\usepackage{kaotheorems}

% Load the package for hyperreferences
\usepackage{kaorefs}

\graphicspath{{images/}{./}} % Paths where images are looked for

\makeindex[columns=3, title=Alphabetical Index, intoc] % Make LaTeX produce the files required to compile the index


\begin{document}

%----------------------------------------------------------------------------------------
%	BOOK INFORMATION
%----------------------------------------------------------------------------------------

% \titlehead{Document Template}
\title[Formalizing Coq Modules in the MetaCoq project]{Formalizing Coq Modules in the MetaCoq project}
\subtitle{Bachelor's Thesis}
\author[YJ Tan]{Yee-Jian Tan}
\date{\today}
% \publishers{An Awesome Publisher}

%----------------------------------------------------------------------------------------

\frontmatter % Denotes the start of the pre-document content, uses roman numerals

%----------------------------------------------------------------------------------------
%	COPYRIGHT PAGE
%----------------------------------------------------------------------------------------

\makeatletter
\uppertitleback{\@titlehead} % Header

\lowertitleback{
	% \textbf{Disclaimer} \\
	% You can edit this page to suit your needs. For instance, here we have a no copyright statement, a colophon and some other information. This page is based on the corresponding page of Ken Arroyo Ohori's thesis, with minimal changes.
	
	% \medskip
	
	% \textbf{No copyright} \\
	% \cczero\ This book is released into the public domain using the CC0 code. To the extent possible under law, I waive all copyright and related or neighbouring rights to this work.
	
	% To view a copy of the CC0 code, visit: \\\url{http://creativecommons.org/publicdomain/zero/1.0/}
	
	% \medskip
	
	% \textbf{Colophon} \\
	% This document was typeset with the help of \href{https://sourceforge.net/projects/koma-script/}{\KOMAScript} and \href{https://www.latex-project.org/}{\LaTeX} using the \href{https://github.com/fmarotta/kaobook/}{kaobook} class.
	
	% \medskip
	
	% \textbf{Publisher} \\
	% First printed in May 2019 by \@publishers
}
\makeatother

%----------------------------------------------------------------------------------------
%	DEDICATION
%----------------------------------------------------------------------------------------

% \dedication{I dedicate this thesis to my mother, my sister, and my father; whom
% gave me the wings to fly, and kept my feet on the ground. }

\dedication{ Try not to focus on the logical details, instead, try to focus on
the simplicity behind the details.  \flushright--- Leung Man Chun }

%----------------------------------------------------------------------------------------
%	OUTPUT TITLE PAGE AND PREVIOUS
%----------------------------------------------------------------------------------------

% Note that \maketitle outputs the pages before here
\maketitle

%----------------------------------------------------------------------------------------
%	PREFACE
%----------------------------------------------------------------------------------------
\begingroup
  %in the header, chapters start on any page to avoid too many blank pages
  \let\cleardoublepage\clearpage
  \pagelayout{margin}
\chapter*{Abstract}

This is an abstract. I am formalizing Coq modules in MetaCoq, it is a missing
piece of the already established MetaCoq project. In the second part, as a
requirement for the mathematics requirement, I study some basic properties of
lambda calculus, hopefully culminating in the decidability for conversion in
real-world type systems such as the Martin-Lof Type Theory (MLTT).

\chapter*{Acknowledgements}

I wouldn't come this far with the help of many people along the way. I would
like to thank especially Nicolas Tabareau for having me in Nantes for the summer
of 2022, which changed my life. Thank you professors Yang Yue and Martin Henz of
the National University of Singapore for being my advisors for this project,
even though it a rare one --- externally proposed, cross-faculty, joint final
year project.

I would also like to thank the people who inadvertently made this possible ---
Thomas Tan Chee Kun who gave me a foray into the field of Type Theory, then
Rodolphe Lepigre who was an amazing mentor landing me into a summer of internship.
Not forgetting the valuable friendship and advice I received in Gallinette team,
for the people whom I interacted with: Pierre, Tomas, Yann, Arthur, Iwan, Kenji,
Meven, Assia, Pierre-Marie, Hamza, Enzo, Yannick and everyone else that appeared
during my stay in Gallinette. I truly loved the time there and can't wait to be
back again.

\chapter*{Reading this paper}

This paper is written with screen readers in mind. I will add links and
references wherever possible, especially in any upcoming definitions. Click on 
symbols to jump to its definition, if I figure out how to make it work.

\endgroup

%----------------------------------------------------------------------------------------
%	TABLE OF CONTENTS & LIST OF FIGURES/TABLES
%----------------------------------------------------------------------------------------

\begingroup % Local scope for the following commands

% Define the style for the TOC, LOF, and LOT
%\setstretch{1} % Uncomment to modify line spacing in the ToC
\hypersetup{linkcolor=.} % Uncomment to set the colour of links in the ToC
\setlength{\textheight}{230\vscale} % Manually adjust the height of the ToC pages

% Turn on compatibility mode for the etoc package
\etocstandarddisplaystyle % "toc display" as if etoc was not loaded
\etocstandardlines % "toc lines as if etoc was not loaded
\setcounter{tocdepth}{\subsectiontocdepth}

\tableofcontents % Output the table of contents

% \listoffigures % Output the list of figures

% Comment both of the following lines to have the LOF and the LOT on different pages
\let\cleardoublepage\bigskip
\let\clearpage\bigskip

% \listoftables % Output the list of tables

\endgroup


%----------------------------------------------------------------------------------------
%	MAIN BODY
%----------------------------------------------------------------------------------------

\mainmatter % Denotes the start of the main document content, resets page numbering and uses arabic numbers
\setchapterstyle{kao} % Choose the default chapter heading style

\pagelayout{wide} % No margins
% \addpart{Part I\@: Formalization of Coq Modules}
\pagelayout{margin} % Restore margins

% remove inter-chapter blank pages
\let\cleardoublepage\clearpage

% generic intro
\chapter{Introduction}

The Coq Proof Assistant is one of the prominent proof assistants, with
applications from the Homotopy Type Theory (HoTT)'s univalent foundation of
mathematics, to the Iris proof mode for Concurrent Separation Logic, or the
recent receipient of the 2021 ACM Software System Award --- the CompCert
compiler.  However, even though the theory behind Coq, the Calculus of Inductive
Constructions is known be consistent and strongly normalizing (hence
proof-writing is decidable), but the OCaml implementation of Coq is known to
have an average of one critical bug per year which allows one to prove False
statements in Coq.

The MetaCoq project is therefore started by the Galinette Team in INRIA, to 
"formalize Coq in Coq" and acts as a platform to interact with Coq's terms
directly, in a verified manner. This also reduces the trust in the
implementation of Coq to the correctness of the theories underlying Coq. In
2020, the core language of Coq, minus a few features are already successfully
verified in Coq. However, there are still a few missing pieces not yet verified,
among them the Module system of Coq.

The Module system, although not part of the core calculus of Coq, is an
important feature for Coq developers to develop modularly, providing massive
abstraction and a suitable interface for reusing definitions and theorems.

In this project, I aim to formalize the Module system of Coq using the MetaCoq
project framework, by first understanding the implementation of Modules, and
then providing a specification of the implementation of Coq modules, finally 
writing proofs to show the correctness of the current implementation.
In particular, I will focus on the formalization of non-parametrized, plain
modules.

In order to tackle this project, it is important to understand the theory behind
the implementation of Coq; more explicitly, how Type Theory helps in theorem 
proving. This report is therefore structured from the bottom up, from some
results of Type Theory which I study in under the supervision of Professor Yang
Yue; to the formalization of Modules jointly supervised by Professor Nicolas 
Tabareau and Professor Martin Henz.

The first section will give an overview of Type Theory and Theorem Proving.
The second section will cover some important results from Type Theory that I
studied. Following that, I will give an introduction to the language of Coq and
the core problem of formalizing Coq Modules, before describing some related
works and finally a specification for Coq Modules which I will implement.

% we start by explaining how type-theoretic proof assistants work
\chapter{Types and Proof Assistants}
% \section{Types and Proof Assistants}

\section{What are proof assistants?}
% \subsection{What are proof assistants?}
Proof assistants are a special kind of computer program whose job
is to verify the correctness of a mathematical proof, that is,
verify if a conclusion can be made by applying a fixed set
of logical deduction steps on a set of fixed assumptions. Modern 
implementations of proof assistants usually utilizes theories such as 
Higher Order Logic (eg. Isabelle, HOL Light), or some form of Type Theory.

In order to have a computer program that can do such powerful logical
reasoning to contain sophisticated mathematical results, the first 
question that arises is how to represent a theorem and what qualifies
as a valid proof of a theorem. Fortunately, Type Theory gives a simple
and direct relationship between Logic and Types.

\section{Type Theory}
% \subsection{Type Theory}

Types were first invented by Bertrand Russell as a way to provide higher-order
structures than sets to overcome his famous Russell's paradox. However, it was
Alonzo's Church's Lambda Calculus which when used with types (Simply Typed
Lambda Calculus) exhibited many desirable properties as not just a consistent
formal system, also as a rewriting system that exhibits strong normalization.
The arguably most important characteristic of Type Theories to qualify as a 
useful theory for the implementation of proof assistants, lies in the
Curry-Howard Correspondence.

\section{The Curry-Howard Correspondence}
% \subsection{The Curry-Howard Correspondence}

The Curry-Howard Correspondence establishes an one-to-one correspondence between
two distinct disciplines --- Logic and Types. It asserts that there is a
one-to-one correspondences between Types and Propositions, and between Terms and
Proofs. More precisely, it states that every proposition in intuitionistic
propositional logic (constructive 0-th order logic) can be expressed as a Type
in Simply Typed Lambda Calculus, and correspondingly, a term of a specific type
is viewed as a proof for the proposition represented by the type.

\begin{tabular}{c|c}
Intuitioinistic Propositional Logic & Type Theory\\
\hline
Proposition & Type\\
Proof & Term
\end{tabular}

What are those exactly?

\subsection{Intuitionistic Propositional Logic}
% \subsubsection{Intuitionistic Propositional Logic}

Propositional logic, also known as \emph{0-th order logic} is a logical system
where propositions (sentences) are constructed with variables and logical
connectors, such as $\to$ (implication), $\vee$ (disjuction), $\neg$ (negation),
$\wedge$ (conjunction) and more. For example, propositions in Proprositional
logic looks like:

\begin{enumerate}
    \item $p\vee q$
    \item $p\wedge \neg p$
    \item $(p\to q)\to (\neg q\to\neg p)$
\end{enumerate}

One can assign a valuation to each variable, either $1$ or $0$, which represents
"True" or "False" informally. If there is some valuation that result in the
propostion result in $1$, then we say the proposition is \emph{satisfiable}; if 
there is none, i.e. all valuations will result in the proposition being $0$, 
we say the proposition is a \emph{contradiction}; on the other hand, if any 
valuation will result in $1$, the proposition is a \emph{tautology}.

In particular, in the above example, the propositions are satisfiable, a
contradiction and a tautology in that order.

Logical systems can be either intuitionistic or classical, based on one fact:
whether they accept the \emph{rule of the excluded middle}. Precisely, it says

\newtheorem{axiom}{Axiom}[section]
\begin{axiom}[Excluded Middle]
    For any proposition $p$, either $p$ or $\neg p$ holds.
\end{axiom}

Much of modern mathematics is built on classical logic, which asserts the axiom
of excluded middle. This leads to the existence of multitude of non-constructive 
proofs, which shows existence without giving a witness. A classical
non-constructive proof using the law of excluded middle is as follows:

\begin{theorem}
There exists irrational numbers $a, b$ where $a^b$ is rational.
\end{theorem}
\begin{proof}
By the law of excluded middle, $\sqrt{2}^{\sqrt{2}}$ is either rational or
irrational. 
    
Suppose it is rational, then take $a=b=\sqrt{2}$. We have found such $a,b$.

Suppose it is irrational, then take $a=\sqrt{2}^{\sqrt{2}}$, $b=\sqrt{2}$. Then
\[a^b={\left(\sqrt{2}^{\sqrt{2}}\right)}^{\sqrt{2}}=\sqrt{2}^{\sqrt{2}\times\sqrt{2}}=2\]
is rational.
\end{proof}

Even with the proof, we still don't have a concrete pair of irrational numbers
$(a,b)$ fulfilling the property, but the conclusion is declared to be logically
sound.

Intuitionistic logic, on the other hand, rejects this rule, leading to a logical
system where to prove something exists, it requires an algorithm for the
explicit construction of such object.

\subsection{Simply Typed Lambda Calculus}
% \subsubsection{Simply Typed Lambda Calculus}

The Simply Typed Lambda Calculus (STLC) is a simple construction:

\subsection{The Correspondence}
This correspondence is two-way: in particular, we can encode
a proposition as a type, and to write a proof is equivalent to 
finding a term that inhabit that particular type. This reduced the
problem of proof-checking to the problem of finding the type of a term,
or under the Curry-Howard correspondence, whether a proof truly proves
a given proposition.

This fundamental correspondence is elegant, but only limited to 
propositional logic. More complex type systems, such as the
Calculus of Constructions by Theiry Coquand, extend the Curry-Howard
correspondence to higher-order logic, finally allowing the encoding of
most mathematical truth and specifying proofs for theories in those 
systems.


% and the motivation for the metacoq project
\chapter{Introduction to the MetaCoq Project}
% \section{Introduction to the MetaCoq Project}

\section{The MetaCoq Project}
% \subsection{The MetaCoq Project}
MetaCoq is a project to formalize the core calculus, PCUIC, in Coq, and become
a platform to write tools that can manipulate Coq terms. The effort was complete
for a large part of the core language of Coq, with a few missing pieces:

\begin{itemize}
    \item Eta 
    \item Template Polymorphism
    \item SProps
    \item Modules
\end{itemize}

I will be tackling the last.

\section{Structure of the MetaCoq Project} The MetaCoq project is an
ambitious project aiming to provide a verified implementation of Coq, and for
its size, it is reasonably split into a few main components. From the layer
closest to the Coq language to the machine code, we have: TemplateCoq (Section
\ref{sec:mc-template}), PCUIC (Section \ref{sec:mc-pcuic}), followed by Safe
Checker, Erasure and beyond(Section \ref{sec:mc-beyond}).

Let us remind ourselves of the task of MetaCoq: we would like to see that the
OCaml representation of Coq is indeed correct and preserves the desired properties 
of the underlying theory. Since Coq has added many bells and whistles for its 
users, the terms of Coq definitely is much more complex than its underlying,
platonic type theoretical form. Therfore, MetaCoq has several stages for a term
to go through, stripping down to the bare minimum through the following few 
stages.

\subsection{TemplateCoq}
\label{sec:mc-template}

TemplateCoq is a quoting library for Coq: a Coq program that takes a Coq term,
and constructs an inductive data type that correspond to its kernel
representation in the OCaml implementation. This is the first layer of the
stripping of a Coq term, where the structures are preserved properly.

% TODO: Insert example

This allows one to turn a Coq program into a Coq internal representation along
with its associated environment structures, such as the definitions and
declarations in the environment.

\subsection{PCUIC}
\label{sec:mc-pcuic}

PCUIC is the Polymorphic Cumulative Calculus of Inductive Constructions. It is a
"cleaned up version of the term language of Coq and its associated type system,
shown equivalent to the one in Coq." (From MetqCoq website). In other words, it 
is a type theory that is as powerful as Coq can express, having the good
properties such as weakening, confluence, principality (that every term has a
principal type) etc.

A term generated in TemplateCoq can be converted into PCUIC term via a verified
process. The theory of PCUIC is then proven to have all the nice properties in
Coq.


\subsection{Safe Checker, Erasure and Beyond}
\label{sec:mc-beyond}

The core semantic operation of type theories are the reductions. The safechecker
is a verified "reduction machine, conversion checker and type checker" for PCUIC
terms. So far, we have the tools to start with a Coq term, first quoting into
TemplateCoq, then converted into a PCUIC term, and eventually has its type
checked in the Safe Checker via a fully verified process. As far as correctness
is concerned, this has already formed a verified end-to-end process of Coq's 
correctness.

The MetaCoq has further provided a verified Type and Proof erasure process from
PCUIC to untyped Lambda Calculus. The equivalence of this erased language is
can be evaluated in \emph{C-light} semantics, the subset of C accepted by
the CompCert verified compiler, which completes a maximally safe evaluation
toolchain for the language of Coq \sidecite{coqcoqcorrect}.


% then we explain the structure of coq
\chapter{The Coq Proof Assistant}
% \section{The Coq Proof Assistant}

The Coq Proof Assistant (or Coq for short) is a proof assistant 
based on the Calculus of Inductive Constructions, which added
co-inductive types into the Calculus of Constructions.

\section{An example of a Coq program} This is a simple example featuring the
definition of a type ``nat'', a definition of a term of type ``nat'', namely
``zero'', and a function ``plus''  which takes two ``nat''s and returns its sum.

\begin{listing}[!ht]
\inputminted{coq}{code/example.v}
\caption{A simple Coq program.}
\end{listing}

Finally, as explained before, a proposition is a type; and here a named
proposition is postulated. A simple proof is also given, which constructs a term
inhabiting that type, hence proving the correctness of this proposition.

\section{Coq Modules}
Modules in Coq not only allows the reuse of code, it also provides parametrized
theories or data structures in the form of functors (or parametrized modules).

In the following example, we show how one can package definitions into named modules
for reuse; we also show how we can create parametrized generic data structures.

\begin{listing}[!ht]
\inputminted{coq}{code/module_example.v}
\caption{An example of Modules.}
\end{listing}

In the final line, a new Magma was created by the functor ``DoubleMagma''.
% what is this project about? spotlight on modules in coq and how it works
\section{Towards Specifying the Meaning of Coq Modules}

% \subsection{A specification of Coq Modules}
Module systems are a feature of the ML family of languages; it allows
for massive abstraction and the reuse of code. In particular, Coq 
also has a module system that is influenced by ML modules, first 
implemented by Jacek Chrząszcz in 2003, then modified by Elie Soubrian
in 2010.

There are a few keywords when it comes to Coq Modules:
\begin{itemize}
\item A \textbf{structure} is an anynomous collection of definitions, and is the 
    underlying construct of modules. They contain \textbf{structure elements},
    which can be a
    \begin{itemize}
    \item constant definition
    \begin{minted}{coq}
        Definition a: bool := true.
    \end{minted}
    \item assumption
    \begin{minted}{coq}
        Axiom inconsistent: forall p: Prop, p.
    \end{minted}
    \item \textbf{module, module type, functors} recursively.
    \end{itemize}

\item A \textbf{module} is a structure given a name. It can be defined explicitly to be
    of a certain \textbf{module type}, which a named structure with possibly
    empty definitions.
\item A \textbf{module alias} is the association of a short name to an existing
    module.
\item A \textbf{functor} is a module defined with a parameter with a binder and
    a required type for the module supplied as an argument.
\end{itemize}

For a more precise definition of modules and related structures, please refer to
\href{https://coq.inria.fr/refman/language/core/modules.html}{Coq: Modules}.

\subsection{Conversion of Coq Terms}

To understand Coq Modules, we need to first understand the basic structure of
Coq. The core object in the language of Coq are terms. Terms of a type
correspond to a proof for a theorem as in the Curry-Howard correspondence.  The
syntax and semantics of Coq terms are as explained by the syntax,
\sidenote{\href{https://coq.inria.fr/refman/language/core/basic.html\#essential-vocabulary}
{Coq: Essential Vocabulary}} conversion (including reduction and expansion)
\sidenote{\href{https://coq.inria.fr/refman/language/core/conversion.html} {Coq:
Conversion}} and typing
\sidenote{\href{https://coq.inria.fr/refman/language/cic.html} {Coq: Typing}}
respectively. The evaluation of Coq terms are done under a Global Environment
$\Sigma$ containing definitions, and a local context $\Gamma$ containing
assumptions. Evaluation in Coq is known as conversion, the reflexive, transitive
closure of the various reduction rules that is defined, including the famous
$\beta$-reduction (function application).

The conversion relation is then defined with these parameters, from which we can
conclude nice properties on conversion, such as strong normalization, confluence
and decidability.

\subsection{Modules as second-class objects}
% \subsubsection{Modules as second-class objects}
However, Coq modules are not first class objects of the language and do not
participate in conversion themselves; i.e. there is no notion to "reduce" a
module. Plain modules in Coq can be treated as a named container of constant
and inductive definitions, including possibly nested modules; namespaced by a 
dot-separated string called a "path". This abstraction allows users to reuse
definitions, essentially importing another "global environment" into the current
one.

To further expand this possibility, module functors exist to be interfaces which
users can provide definitions for, by supplying a module definition. Functors
are therefore opaque second-class objects which is only useful when a module is
generated.

In this chapter, we formalize the semantics of the current implementation of Coq
Modules and formalize them at the level of TemplateCoq. Section
\ref{sec:semantics-of-modules} describes the semantics behind plain modules and
aliased modules, without functors.


% \section{Related Works}
\subsection{Related Works} Jacek Chrzaszcz's article in TPHOLs 2003
\sidecite{jacek2003} explains the motivation and choices for the implementation
of modules as a second class object and its interaction with terms as described
above.  Elie Soubiran's PhD Thesis on Coq Modules \sidecite{soubiran} describes
an extension of the Module system which only some features are implemented. The
\href{https://github.com/coq/coq/wiki/ModuleSystem}{ModuleSystem Wiki Page} on
Coq's official Github repository contains valuable information on the usage and
design decision of the Module system of Coq, including a list of open issues 
with Modules. In general, the issues do not compromise the correctness of the
type system implemented by Coq; instead, they should be viewed as possible areas
of improvements for Coq users.

Other slightly useful references include papers that explain modules in the ML
family of languages, specifically OCaml and to a certain extent, Standard ML.
On this note, Derek Dreyer wrote his PhD thesis \sidecite{dreyerphd} on
understanding and extending ML modules, and subsequently on implementing ML
modules in its most desirable form, applicative and first-order as a subset of a
small type system, $F_\omega$\sidecite{f-ing}. Other notable implementations
include that of CakeML \sidecite{cakeml}, a verified ML language and Standard ML
modules by MacQueen\sidecite{macqueen1984modules}. However, since the type
system of Coq is much stronger and sophisticated compared to ML languages, the
implementations also vary wildly and one can only refer to them for
inspirations.


% \begin{enumerate}
% \item Elie Soubrian's PhD Thesis on Coq Modules is the latest reference on 
% Coq Modules specifically.
% \item Since ML modules are the source of
% inspiration for Coq's Modules, it is also worthwhile to study
% Derek Dreyer's PhD Thesis on "Understanding and Evolving the ML Module 
% System".
% \item Recently, F-ing modules by Dreyer et. al. provides an 
% implementation of the ML module system in plain $F_\omega$, a strict 
% subset of CIC and hence PCUIC used in modern Coq.
% \item Standard ML also have modules.
% https://homepages.inf.ed.ac.uk/mfourman/teaching/mlCourse/notes/sml-modules.html
% Michael Fourman has lecture notes.
% \item David Macqueen (2002) wrote about implementint SML modules.
% \item CakeML has modules, but not functors. YK Tan (2015).
% \end{enumerate}

\section{Semantics of Modules}
\label{sec:semantics-of-modules}
From now onwards, we consider only non-parametrized modules.

There are two operations involving modules: how to define a module and how to
use a module. We will specify the behaviour, implementation and proof
obligations below.

Modules are containers for definitions that allow reuse. Definitions in Coq are
stored in a Global Environment. We first look at the structure of Global Environment:

\subsection{Global Environment}
The Global Environment in Coq can be understood as a table or a map. There are
three columns in the map: first is a canonical kername, second a pathname, and
finally, the definition object. Canonical kernames can be though of as unique
labels, and for the ease of understanding, as natural numbers 1, 2, 3 etc.
The pathname is a name which the user gives to the definition; it is of the form
of a dot-separated string, such as $M.N.a$. Finally, the definition object can
be:
\begin{itemize}

\item A \textbf{constant definition} to a Coq term, such as a lambda term,
application term, etc..

\item An \textbf{inductive definition} of a type.

\item A \textbf{module definition}. We consider a module to be inductively
defined as a list of constant, inductive, module or module signature
definitions. Alternatively, it can also be an alias to a previously defined
module (which may be an alias).

\item A \textbf{module signature definition} has the same structure as a module
definition, but instead of concrete definitions, it only specifies a name and a
type for each entry. It can be also an alias to a previously defined module
signature (which may be an alias).

\end{itemize}
The terms ``module type'' and ``module signature'' are used interchangeably.

\subsection{Plain Modules}
\label{sec:plainmodules}

\subsubsection{Behaviour and Implementation}

Modules can be thought of as a named global environment where the definitions
within it are namespaced by the module name. Therefore, its contents are not
modified during conversion/reduction.  In Coq, modules are second-class objects;
in other words, a module is not a term.  Its definition is stored and referred
to by a pathname and a kername.

Therefore, implementation wise, one need to ensure the correctness of "referring
to definition"; that is, when a definition within a module is referred to by its
pathname $M.N.a$, it will be fetched correctly from the table.

\subsubsection{Proof obligation}

We say the implementation of such a module is correct if the meta-theory of the
original system are unchanged and remains correct; that is the proofs go through
when terms can be defined within modules. Since the MetaCoq project has proven
various nice properties about conversion in Coq, our project on plain modules is
two-fold:

\begin{enumerate}
\item Ensure the correctness of the static semantics of Coq Modules
(well-typedness) during its definition.
\item Define the behaviour of access of definitions within Modules.
\end{enumerate}

Once these two are done, we can be sure that a Coq program with Modules
has all its terms well-defined (by (1)) and enjoys the nice properties of
conversion, since the additional terms defined in Modules fulfill (2).
This follows as our definition of Modules on the TemplateCoq level, is
eventually elaborated down into the PCUIC calculus the idea of modules and
aliasing do not exist anymore, they are flattened into the corresponding global
environment. The details of (2) are described in Section \ref{sec:using-modules}.

Concretely, if a module as below is defined while the Global Environment, which
stores definitions is denoted as $\Sigma$:
\begin{minted}{coq}
Module M.
    Definition a: nat := 0.
End M.
\end{minted}
Then the environment must have a new declaration added:
\[\Sigma := \Sigma :: \text{ModuleDeclaration}(M,
[\text{ConstantDeclaration}(M.a, nat, 0)])\]

So when $M.a$ is called, it must refer to the definition in the Global
Environment correctly.

\subsection{Aliased Modules}
Aliased modules are just a renaming of existing modules, which can be seen as
syntactic sugar for modules. Therefore, the correctness depends only on
implementing this internal referencing correctly.

\subsubsection{Behaviour and Implementation}
Suppose we have

\begin{minted}{coq}
Module N := M.
\end{minted}

Aliasing $N$ to $M$ and $M$ is a previously defined module (or an alias),
then any access path $N.X$ should be resolved similarly to
$M.X$ (note that since $M$ is possibly an alias as well, we do not require $N.X$
to resolve \emph{to} $M.X$). In the OCaml implementation, all definitions in $M$
can now be referred to by the pathnames $N.X$ in addition to $M.X$, while still
having the same kername.

\subsubsection{Proof Obligations}
\begin{enumerate}
\item Well-definedness: aliasing can only occur for well-defined modules. There
    cannot be self-alias and forward aliasing (aliasing something not yet defined).
\item The resolution of aliased modules is done at definition. If $N$ is aliased
    to $M$, then $N$ will immediately inherit the same kername as $M$. We will
    show this resolution is decidable and results in correct aliasing.
\end{enumerate}

\subsection{Using Modules}
\label{sec:using-modules}
As mentioned, the only way modules are used is during reduction or conversion
of a Coq term. In Coq, reduction and conversion are made up of smaller reduction
rules, such as $\beta, \delta,\zeta,\eta,\iota$ reductions. In particular, 
Modules are related only to $\delta$ reductions, which "replaces a defined
variable with its definition" 
\sidenote{\href{https://coq.inria.fr/refman/language/core/conversion.html} {Coq:
Conversion}}.

The correctness of $\delta$-reduction and conversion is a meta-theoretic
property, which is already shown to be correct and have properties such as 
normalization, confluence etc. in PCUIC \sidecite{coqcoqcorrect}. I will
contribute by expanding the definition of delta-conversion and expand the
existing proofs in the MetaCoq project that such properties continue to hold.

% Finally: what is the plan for next semester
\chapter{Project Plan for The Next Semester}
% \section{Project Plan for The Next Semester}

Since this is a joint thesis with the Mathematics department, also due to 
personal interest, for the remaining semester for this thesis, I am planning to:

\section{Implement Plain Coq Modules}
There are a few steps to this implementation that I can forsee:
\begin{enumerate}
\item Study the structure and conventions of the MetaCoq project. Since this is 
    a huge project, many common theorems have been abstracted into multiple
    functors which can generate a theorem given a corresponding object, such as
    % TODO: add functor example
\item Add the data structure of Plain Modules into the global environment which
    is quoted by TemplateCoq. This requires the modification of the current data
    structure containing declarations (minus modules). 
\item Ensure the environment remains well-defined with the current typing rules
    and correctness assertions. This requires the modifications to the existing 
    proofs on the environment (well-formedness, well-typedness).
\item Extend the TemplateCoq plugin (in OCaml) which transforms a Coq term into 
    a TemplateCoq representation, to translate a Coq module into the internal
    representation of Modules I implemented above. This requires OCaml knowledge
    specifically in the area of Coq plugins.
\item Now that syntatical transformation from Coq to TemplateCoq is complete, I
    need to define how these modules that live in the Global Environment will 
    be used. In particular, I need to implement the "canonical name" model as
    implemented in the Coq kernel.
\item Finally, the translation from TemplateCoq to PCUIC. In PCUIC,
    modules will cease to exist and references to definitions in modules will be
    treated as direct references to plain definitions outside modules. This is
    done by referring module paths to their canonical paths.
\item The last step in this project is to verify that the translation from
    TemplateCoq to PCUIC (and vice versa) is correct; that is, a PCUIC term is 
    well-formed and well-typed iff its translation is well-formed and well-typed
    in TemplateCoq.
\end{enumerate}

\section{Study Type Theoretic Results}

The modern field of Type Theory and Proof Assistants grow very quickly, with
many new fields of logic (modal logic, linear logic, seperation logic) etc and
their corresponding Type Theory being studied and made into proof assistants to 
verify results in these fields. Since there were not many relevant courses in my 
undergraduate curriculum, I decide to continue to study some classical results
alongside with the formalization project above.

The direction which I have chosen is towards a result important for the
implementation of proof assistants: the decidability of conversion
(correspondingly proof-checking) in a simpler type system, such as the Martin-Lof
Type Theory. So far, I have covered results from Simply Typed Lambda Calculus
including confluence and weak normalization, and its counterparts in proof theory,
namely sequent calculus, natural deduction as a normal form, and cut-elimination.

% \pagelayout{wide} % No margins
% \addpart{Part II\@: Decidability of Conversion in MLTT}
% \pagelayout{margin} % Restore margins

\appendix % From here onwards, chapters are numbered with letters, as is the appendix convention

\pagelayout{wide} % No margins
\addpart{Appendix}
\pagelayout{margin} % Restore margins

\chapter{Specification of Coq Modules}
% \section{Specification of Coq Modules}

\section{Typing Rules}

I will be typesetting the rules here, taken from \href{https://coq.inria.fr/refman/language/core/modules.html#typing-modules}{Coq: Typing Modules}.
\chapter{The Untyped Lambda Calculus}
% \section{The Untyped Lambda Calculus}

Here, I present some properties of the untyped lambda calculus I studied.  We
follow the text "Proof and Types" by Girard. Here, I present some fundamental
results.

\section{The Church-Rosser Confluence}
% \subsection{The Church-Rosser Confluence}

We study the paper by Takahashi's successors on a proof for Church-Rosser 
theorem. It states that:
\begin{theorem}[Church-Rosser]
If $M\to_\beta M_1$, $M\to_\beta M_2$, then there exists a term $N$ such that
$M_1\to_\beta N$ and $M_2\to_\beta N$.
\end{theorem}

This property asserts the uniqueness of a normal form of a rewriting system, if
there exists one. This is especially important for systems that are known to
normalize; in particular, the Simply Typed Lambda Calculus, the Martin Lof Type
Theory, and Calculus of Constructions are all known to be strongly normalizing.
Together with the decidability of conversion, we have a viable theory for 
the implementation of proof assistants.

The crucial step of the proof is to define a notion of parallel reduction, and
the key lemma is

\begin{lemma}
If $M\to_{\beta n}N$, then $M\to_{\beta}M^{n*}$.
\end{lemma}

This provides a candidate of confluence just based on the original term $M$,
instead of the intermediate steps during $M\to_\beta M_1,M_2$ or even $M_1, M_2$
itself.

(Hopefully I have a Coq proof.)

\section{Cut elimination}
% \subsection{Cut elimination}

Sequent Calculus is a logical system capable of doing deduction. One of the 
axioms, the Cut rule, is actually eliminable. This can be proven by induction.

(I will insert the proof here).

%----------------------------------------------------------------------------------------

% \backmatter % Denotes the end of the main document content
% \setchapterstyle{plain} % Output plain chapters from this point onwards

%----------------------------------------------------------------------------------------
%	BIBLIOGRAPHY
%----------------------------------------------------------------------------------------

% The bibliography needs to be compiled with biber using your LaTeX editor, or on the command line with 'biber main' from the template directory

\defbibnote{bibnote}{Here are the references in citation order.\par\bigskip} % Prepend this text to the bibliography
\printbibliography[heading=bibintoc, title=Bibliography, prenote=bibnote] % Add the bibliography heading to the ToC, set the title of the bibliography and output the bibliography note

%----------------------------------------------------------------------------------------
%	INDEX
%----------------------------------------------------------------------------------------

% The index needs to be compiled on the command line with 'makeindex main' from the template directory

\printindex % Output the index

\end{document}