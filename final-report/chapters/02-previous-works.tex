\chapter{Previous Works}

Since Coq is influenced by the ML family of languages, the specification for
modules in Coq is very similar to that of OCaml, and Standard ML (SML). in this
chapter, we review previous implementation of Coq modules, as well as relevant
module systems in other ML languages that this project can refer to.

% To implement modules, we can study previous implementation of modules in both
% Coq and related languages from the ML Family, such as Standard ML (SML) and
% OCaml.

% Things to explain: type system, mod sig fun, anonymous second class obj
% conservative ext

\section{Coq Module Implementations}
The earliest exploration of adding a module system to Pure Type System (PTS)
\sidenote{PTS can be seen as a generalization of Barendregt's Lambda Cube,
by defining type systems using a triple $(\mathscr{S},\mathscr{A},\mathscr{R})$
represnting sorts, axioms and rules of the type system respectively.},
a generalized type system, was done by Judicaël Courant. He designed the $MC$
Module Calculus system that includes modules, signatures, and functors for PTS
and proved that it is a conservative extension.
\sidenote{In proof theory, a conservative extension of a formal language is one
that cannot prove statements that are not already provable in the base language.}
 Modules in $MC$ are 
anonymous, second-class objects with a specific set of reduction rules, and
Courant has proven the resulting system to have decidable type inference and 
the principal type property.

Building on the idea of Courant, Jacek Chrąszcz designed the
earliest implementation of a module system in Coq in his PhD thesis, and was
released with Coq version $7.4$. The module system by Chrąszcz was a subset of
that of Courant with some changes. Similar to $MC$, modules, signatures and
functors are implemented together with specialized reduction rules,
but he argued that an anonymous module system does not work well with the
definition and rewriting system of Coq. Therefore, all modules are named in
Chraszcz's implementation, and the expression of modules
in Coq is restricted only to module paths. The core of Chrąszcz's PhD thesis is
the conservativity proof about the module system extension over Coq, together
with the syntax, typing rules, and rewriting rules of the module system in Coq.

The most recent work on Coq's module system is Elie Soubrian's PhD thesis. He
proposed many improvements on the system, among which,
a unified notion of structure for modules and signatures, and the availability of
higher-order functors are already in today's OCaml implementation of Coq.
However, other features mentioned in the thesis, such as applicative functors,
and a notion of namespace that allow a separate, dynamic naming scope for modules,
are not yet implemented in the module system in Coq today.

\section{Modules in ML Dialects}
The two main dialects ML today, OCaml and Standard ML have interestingly
different semantics for modules. Modules are by default applicative (generative
functors are possible) in OCaml while generative in Standard ML.

The module system of SML has evolved over the years, from the earliest account
by MacQueen\sidecite{macqueen1984modules} and Harper et. al. in terms of "strong sum"
types, to the "transparent" approach by Lillibridge. Harper and Lillibridge also
developed first-order modules in SML eventually using standard notions from type
theory. Meanwhile, Leroy made progress on applicative functors, modular module
systems, and mutually recursive modules in SML, then OCaml.

On this note, Derek Dreyer wrote his PhD thesis \sidecite{dreyerphd} on
understanding and extending ML modules, and subsequently on implementing ML
modules in its most desirable form, applicative and first-order as a subset of a
small type system, $F_\omega$\sidecite{f-ing}. Another related project is 
CakeML \sidecite{cakeml}, a verified subset of the SML language, but
unfortunately does not include the verification of the module system.

However, since the type system of Coq
is much stronger and sophisticated compared to ML languages, the
implementations also vary wildly and one can only refer to them for
inspirations.

\section{Knowledge Management in Proof Assistants}

% \begin{enumerate}
% \item Elie Soubrian's PhD Thesis on Coq Modules is the latest reference on 
% Coq Modules specifically.
% \item Since ML modules are the source of
% inspiration for Coq's Modules, it is also worthwhile to study
% Derek Dreyer's PhD Thesis on "Understanding and Evolving the ML Module 
% System".
% \item Recently, F-ing modules by Dreyer et. al. provides an 
% implementation of the ML module system in plain $F_\omega$, a strict 
% subset of CIC and hence PCUIC used in modern Coq.
% \item Standard ML also have modules.
% https://homepages.inf.ed.ac.uk/mfourman/teaching/mlCourse/notes/sml-modules.html
% Michael Fourman has lecture notes.
% \item David Macqueen (2002) wrote about implementint SML modules.
% \item CakeML has modules, but not functors. YK Tan (2015).
% \end{enumerate}
