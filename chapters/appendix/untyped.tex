\chapter{The Untyped Lambda Calculus}
% \section{The Untyped Lambda Calculus}

Here, I present some properties of the untyped lambda calculus I studied.  We
follow the text "Proof and Types" by Girard. Here, I present some fundamental
results.

\section{The Church-Rosser Confluence}
% \subsection{The Church-Rosser Confluence}

We study the paper by Takahashi's successors on a proof for Church-Rosser 
theorem. It states that:
\begin{theorem}[Church-Rosser]
If $M\to_\beta M_1$, $M\to_\beta M_2$, then there exists a term $N$ such that
$M_1\to_\beta N$ and $M_2\to_\beta N$.
\end{theorem}

This property asserts the uniqueness of a normal form of a rewriting system, if
there exists one. This is especially important for systems that are known to
normalize; in particular, the Simply Typed Lambda Calculus, the Martin Lof Type
Theory, and Calculus of Constructions are all known to be strongly normalizing.
Together with the decidability of conversion, we have a viable theory for 
the implementation of proof assistants.

The crucial step of the proof is to define a notion of parallel reduction, and
the key lemma is

\begin{lemma}
If $M\to_{\beta n}N$, then $M\to_{\beta}M^{n*}$.
\end{lemma}

This provides a candidate of confluence just based on the original term $M$,
instead of the intermediate steps during $M\to_\beta M_1,M_2$ or even $M_1, M_2$
itself.

(Hopefully I have a Coq proof.)

\section{Cut elimination}
% \subsection{Cut elimination}

Sequent Calculus is a logical system capable of doing deduction. One of the 
axioms, the Cut rule, is actually eliminable. This can be proven by induction.

(I will insert the proof here).