\chapter{Introduction}

The Coq Proof Assistant is one of the prominent proof assistants, with
applications from the Homotopy Type Theory (HoTT)'s univalent foundation of
mathematics, to the Iris proof mode for Concurrent Separation Logic, or the
recent receipient of the 2021 ACM Software System Award --- the CompCert
compiler.  However, even though the theory behind Coq, the Calculus of Inductive
Constructions is known be consistent and strongly normalizing (hence
proof-writing is decidable), but the OCaml implementation of Coq is known to
have an average of one critical bug per year which allows one to prove False
statements in Coq.

The MetaCoq project is therefore started by the Galinette Team in INRIA, to 
"formalize Coq in Coq" and acts as a platform to interact with Coq's terms
directly, in a verified manner. This also reduces the trust in the
implementation of Coq to the correctness of the theories underlying Coq. In
2020, the core language of Coq, minus a few features are already successfully
verified in Coq. However, there are still a few missing pieces not yet verified,
among them the Module system of Coq.

The Module system, although not part of the core calculus of Coq, is an
important feature for Coq developers to develop modularly, providing massive
abstraction and a suitable interface for reusing definitions and theorems.

In this project, I aim to formalize the Module system of Coq using the MetaCoq
project framework, by first understanding the implementation of Modules, and
then providing a specification of the implementation of Coq modules, finally 
writing proofs to show the correctness of the current implementation.
In particular, I will focus on the formalization of non-parametrized, plain
modules.

In order to tackle this project, it is important to understand the theory behind
the implementation of Coq; more explicitly, how Type Theory helps in theorem 
proving. This report is therefore structured from the bottom up, from some
results of Type Theory which I study in under the supervision of Professor Yang
Yue; to the formalization of Modules jointly supervised by Professor Nicolas 
Tabareau and Professor Martin Henz.

The first section will give an overview of Type Theory and Theorem Proving.
The second section will cover some important results from Type Theory that I
studied. Following that, I will give an introduction to the language of Coq and
the core problem of formalizing Coq Modules, before describing some related
works and finally a specification for Coq Modules which I will implement.