\chapter{Introduction}

\section{Why Theorem Provers?}
Deductive, logical proofs have long been a way for humans to deduce facts and
truths. However, after several centuries of development, mathematics have grown
so huge that mathematicians have no choice but to depend on previous results to
develop new theorems. This requires a form of trust in correctness of
pen-and-paper proofs of previous results, which in turn might be dependent on
previously proved, or even facts dismissed as "obvious". Theoretically,
mathematicians know that every theorem can be boiled down to merely logical
consequences from pre-determined axioms, but as more and more recent proofs are
found to be erroneous (CHECK Voevodsky) to different extents, the field is now
in need of a mechanical verification.

In the realm of computer programming, another problem arises, which similarly
cries for a similar mechanical verification - the correctness of computer
programs. As Dijkstra said, testing can only show the presence of bugs, but not
the absence of it, to ensure our programs are absolutely true to our intentions,
a formal verification or proof is needed. Being the tool for automation itself,
the act of automatically checking the correctness of programs and the lack of
bugs is, in turn, another task solvable by automation, by programming.

Therefore, proof assistants, or sometimes theorem provers are thus born. In its
core, it takes a mathematical claim, and checks the user's proofs against a
logical system with axioms and deductive rules. However, there are two
approaches to writing such proofs - one can have it automated, thus the family
of Automated Theorem Provers (ATP) and the family of Interactive Theorem Provers
(ITP). Coq is a ITP that implements the Polymorphic Cumulative Calculus of
Inductive Constructions (PCUIC), which can prove facts in higher order logic. It
is among the earliest ITPs of its time and has been used to complete large
proofs such as the Four Color Theorem and the Feit-Thompson theorem, and was
awarded the ACM Software System Award in 2014. In 2021, the CompCert C compiler,
a industrial-strength compiler verified in Coq has been awarded the ACM Software
award, recognizing the potential of ITPs in the real of verifying software
correctness as well.

\section{Why MetaCoq?}

As the ancient saying goes: who watches the watchers? While proof assistants
check for the correctness of proofs and programs (yes, they are equivalent under
the Curry-Howard Correspondence), who checks the correctness of the proof
assistants, which are in turn programs themselves? Indeed, although the theory
behind Coq, PCUIC is known to be mathematically "nice", such as strongly
normalizing (and hence decidable), however, the current implementation of Coq in
OCaml is known to be not entirely bug-free -- at least one critical bug that
threatens the correctness of Coq is found, meaning that it is possible to prove
\emph{False} in Coq!

One way to remedy this situation is to have a verified implementation for Coq
itself - possibly in Coq! The MetaCoq project is in general a metaprogramming
platform for Coq, where users can reify Coq terms in Coq and manipulate them,
and thus giving rise for a perfect environment to argue about various properties
of Coq terms, specifically all the correctness properties of the underlying
PCUIC system. Then, an verified implementation of Coq can be implemented in the
MetaCoq project, and after passing through a series of verified components, give
rise to a compiled, verified version of Coq. Of course, without the low-level
optimizations in OCaml, the MetaCoq implementation is definitely slower than the
original Coq, but it can be an implementation that one runs once every month,
for the highest level of security and assurance.

The careful reader here might have already noticed a problem in verifying a Coq
implementation in Coq - Gödel's Second Incompleteness stands in the way of
proving a formal system's consistency with itself. Noticing this problem, the
approach by the MetaCoq implementation is to assume the consistency of PCUIC as
an axiom, moving the trust from the correctness of the OCaml code to the
correctness of the theory of PCUIC itself.

\section{Why Modules?}

Organizing and generalizing human knowledge is one of the most natural things
humans do when trying to understand this world. From the distinction of
different "subjects" ranging from history to engineering, or in the case of
mathematics, from algebra to statistics. The use of modules is for a similar
purpose in proof assistants: to categorize, package and organize knowledge, and
is deeply related to the idea of modular programming and "programming
in-the-large".

Therefore, even though adding the idea of modules to Coq or any other formal
logic systems should not increase the "power" of the system (in fact, when
extending a formal system with modules, one should actively ensure that the
extension is a conservative one), but it is almost essential for the users and
developers to better organize their proofs. The MetaCoq project has successfully
implemented and verified a large subset of the Coq language in 2020 (?), with
the exception with a few features such as $\eta$-expansion, Module System, Template
Polymorphism and Proof-Irrelevant Propositions. Therefore, when I interned at
the birthplace of the MetaCoq project, the \emph{Gallinette} team in Nantes,
France, I chose to work on implementing Modules in the MetaCoq project.

\section{My Contributions}

In this project, I have implemented a non-parametrized module system in the
MetaCoq project and verified various properties about modules and its
interaction with the global environment of Coq, the typing of Coq terms,
$\eta$-expansion and more. Furthermore, I have written a translation between the
language with (non-parametrized) modules to PCUIC without modules, hence the use
of non-parametrized modules is a subset and hence a conservative extension of
the original language.

In this report, I detail my implementation of non-parametrized modules and the
considerations behind the design choices made. I will also explain the related
correctness properties related to modules which I proved. As a learning project,
I will also detail Coq-specific skills that I learnt during the project.

The rest of the report is structured as follows: I will start by a review of
previous related works about the implementation of Modules in Coq and relevant
systems. Then, I will explain the syntax and semantics of Coq modules, before
describing my implementation of Modules and the verification of its properties,
the core of this project. After that, I will describe a second implementation of
Modules in Coq, which I call the Modular Environment rewrite that solves various
problems surfaced from the initial implementation. Finally, I will document the
Coq-specific, interactive-proving related skills I learnt during the project.
This report will end with possible future work; while some technical details,
such as concrete typing rules, and some type-theoretic results I studied as a
joint thesis with the Mathematics Department, will be put in the appendix.