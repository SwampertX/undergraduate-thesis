\chapter{Introduction}

\section{Why Theorem Provers?}
Deductive, logical proofs have long been a way for humans to deduce mathematical
facts. However, after several centuries of development, mathematics has grown so
huge that mathematicians have no choice but to depend on previous results to
develop new theorems. This requires a form of trust in the correctness of
pen-and-paper proofs of previous results, which in turn might be dependent on
previously proved, or even facts dismissed as "obvious". Theoretically,
mathematicians know that every theorem can be boiled down to merely logical
consequences from pre-determined axioms, but as more and more recent proofs are
found to be erroneous (\cite{voevodsky_2018}) to different extents, the field is
now in need of mechanical verification.

In the realm of computer programming, another problem arises, which similarly
cries for a similar mechanical verification - the correctness of computer
programs. As Dijkstra said, “Program testing can be used to show the presence of
bugs, but never to show their absence!” (\cite{dijkstra1970notes}) To ensure our
programs are true to our intentions, a formal verification or proof is needed.
Being the tool for automation itself, the act of automatically checking the
correctness of programs and the lack of bugs is, in turn, another task solvable
by automation, by programming.

Therefore, proof assistants (or sometimes known as theorem provers) are thus
born. At its core, a proof assistant takes a mathematical claim and checks the
user's proof against a logical system with axioms and deductive rules. However,
there are two approaches to writing such proofs - one can have it automated,
thus the family of Automated Theorem Provers (ATP) and the family of Interactive
Theorem Provers (ITP). Coq is an ITP that implements the Polymorphic Cumulative
Calculus of Inductive Constructions (PCUIC) (\cite{pcuic2017timany}), which can
prove facts in higher-order logic. It is among the earliest ITPs of its time and
has been used to complete large proofs such as the Four Color
Theorem (\cite{gonthier2008four}) and the Feit-Thompson
Theorem (\cite{gonthier2013machine}), and was awarded the 2013 ACM Software System
Award (\cite{herbelin_2014}). In 2021, the CompCert C compiler, an
industrial-strength compiler verified in Coq has been awarded the ACM Software
award (\cite{acm_award_2022}), recognizing the potential of ITPs in the real of
verifying software correctness as well.

\section{Why MetaCoq?}

As the ancient saying goes: who watches the watchers\footnote{Originally in
latin, \emph{Quis custodiet ipsos custodes?}, literally "Who will guard the
guards themselves?}? While proof assistants check for the correctness of
mathematical proofs and computer programs (which are equivalent as per the
Curry-Howard Correspondence (\cite{howard1980formulae,curry1934functionality})),
who checks the correctness of the proof assistants, which are in turn programs
themselves? Indeed, although the theory behind Coq, PCUIC is known to be
``correct'', such as being strongly normalizing (and hence decidable in type
checking); however, the current implementation of Coq in OCaml is known to be
not entirely bug-free -- at least one critical bug that threatens the
correctness of Coq is found per year, meaning that it is possible to prove
\emph{False} in Coq (\cite{coq_consistency})!

One way to remedy this situation is to have a verified implementation for Coq
itself. The MetaCoq project is in general a metaprogramming platform for Coq,
where users can reify Coq terms in Coq and manipulate them, and thus giving rise
to a perfect environment to argue about various properties of Coq terms,
specifically all the correctness properties of the underlying PCUIC
system (\cite{sozeau2020metacoq}). Then, a verified implementation of Coq can be
implemented in the MetaCoq project, and after passing through a series of
verified components, give rise to a compiled, verified version of Coq. Of
course, without the low-level optimizations in OCaml, the MetaCoq implementation
is slower than the original Coq, but it can be an implementation that one runs
once every month, for the highest level of security and assurance.

The careful reader here might have already noticed a problem in verifying a Coq
implementation in Coq - Gödel's Second Incompleteness stands in the way of
proving a formal system's consistency with itself. Noticing this problem, the
approach by the MetaCoq implementation is to assume the consistency of PCUIC as
an axiom, moving the trust from the correctness of the OCaml code to the
correctness of the theory of PCUIC itself.

\section{Why Modules?}

The verified implementation of Coq under the MetaCoq project has been done for a
substantial, working subset of Coq, excluding the Module system. Why would
implementing modules be a worthwhile effort, or rather, why is the module system
important for proof assistants like Coq, or programming in general?

Organizing and generalizing human knowledge is one of the most natural things
humans do when trying to understand this world. From the distinction of
different "subjects" ranging from history to engineering, or in the case of
mathematics, from algebra to statistics. The use of modules is for a similar
purpose in proof assistants: to categorize, package and organize knowledge, and
is deeply related to the idea of modular programming and "programming
in-the-large". Therefore, even though adding the idea of modules to Coq or any
other formal logic systems should not increase the "power" of the system (in
fact, when extending a formal system with modules, one should actively ensure
that the extension is conservative), but it is almost essential for the users
and developers to better organize their proofs.

The MetaCoq project has successfully implemented and verified a large subset of
the Coq language in 2020, with the exception with a few features such as
$\eta$-expansion, Module System, Template Polymorphism and Proof-Irrelevant
Propositions(\cite{sozeau2020metacoq}). Therefore, when I interned at the
birthplace of the MetaCoq project, the \emph{Gallinette} team in Nantes, France,
I chose to work on implementing Modules in the MetaCoq project.

\section{Contributions}

In this project, I have implemented a non-parametrized module system in the
MetaCoq project and verified various properties of modules and its
interaction with the global environment of Coq, the typing of Coq terms,
$\eta$-expansion and more. Furthermore, I have written a translation between the
language with (non-parametrized) modules to PCUIC without modules, hence the use
of non-parametrized modules is a subset and hence a conservative extension of
the original language.

In this report, I detail my implementation of non-parametrized modules and the
considerations behind the design choices made. I will also explain the related
correctness properties related to the modules which I proved. As a learning
project, I will also detail Coq-specific skills that I learnt during the
project.

The rest of the report is structured as follows: Chapter \ref{ch:previous-works}
will start with a review of previous related works about the implementation of
Modules in Coq and relevant systems. Then, Chapter \ref{ch:coq} will introduce
the MetaCoq project and explain the syntax and semantics of Coq modules, before
describing the implementation of Modules and the verification of their
properties in Chapter \ref{ch:impl1}, the core of this project. After that,
Chapter \ref{ch:impl2} will describe a second possible implementation of Modules
in Coq, called the Modular Environment rewrite that solves various problems that
surfaced from the initial implementation. Finally, Chapter \ref{ch:machineries}
will document the skill acquired in writing formal proofs in Coq during the
implementation of the project. This report will end with possible future work in
Chapter \ref{ch:future}.
% while some technical details, such as concrete typing rules
% % , and some type-theoretic results I studied as a joint
% % thesis with the Mathematics Department
% will be put in the appendix.