\chapter{Tactics and Proof Machineries}
\label{ch:machineries}
Now we discuss some of the main learning points gained during the implementation
of this project. Although these might be common knowledge for the Coq or proof
theory expert, they contain some non-trivial proof-theoretic machinery that is
useful in formal proofs involving inductive data types, even though they might
be dismissed in typical pen-and-paper proofs by their "apparent" correctness.
They are listed below in no particular order.

\begin{itemize}
  \item nested induction, mutual induction, (can always reduce to the case of a
  simple inductive type, but...)
  \item well-founded recursion and measures.
  \item definition of strict partial orders and strengthening of induction
  hypotheses 
  % \item tactics, reflection, proof management.
\end{itemize}

\section{Nested Inductive Types}

Inductive types are a way to define things recursively. A typical example is the
inductive definition of a list:

\begin{listing}[H]
\begin{minted}{Coq}
Inductive list (T: Type) :=
| nil
| cons (t: T) (tl: list).
\end{minted}
  \caption{Definition of a polymorphic list.}
  \label{lst:def-list}
\end{listing}

To argue about lists, we have the following induction principle:

\begin{align*}
  \forall P, (P\ nil) \implies (\forall t\ tl, (P\ tl) \implies P (cons\ t\ tl))
  \implies \forall l, (P\ l)
\end{align*}

Nested inductive types
are inductive types that have nested induction in them, one simple example being
a rose tree (\cite{meertens1988first}):

\begin{listing}[H]
\begin{minted}{Coq}
Inductive roseTree :=
| node (xs: list roseTree).
\end{minted}
  \caption{Definition of a rose tree.}
  \label{lst:def-rosetree}
\end{listing}

where the nested induction occurs at the list of rose trees within a node.
Intuitively, this is a tree structure in which every node contains a "forest" of
trees. Unfortunately, Coq does not generate a strong enough induction principle
for nested inductive types such as the below:

\begin{align*}
  \forall P, (\forall xs, P (node\ xs)) \implies \forall rt, (P\ rt)
\end{align*}

We focus on the induction hypothesis, the condition before the top-level
implication. It omitted the fact that $xs$ is a list, and we need to check each
rose tree within the list with predicate $P$ first. Here is a stronger induction
principle that is generally used:

\begin{align*}
  \forall P, (\forall xs, (\forall x\in xs, P\ x) \implies P (node\ xs))
   \implies \forall rt, (P\ rt)
\end{align*}

Notice the weakening of the induction hypothesis, and thus the strengthening of
the induction principle. In my case, it is the definition of structure fields of
the modular rewrite:

\begin{listing}[H]
\tcc{325}{332}{Environment.v}
  \caption{Definition of structure fields.}
  \label{lst:def-sf}
\end{listing}

where the nested \verb|list (ident × structure_field)| is a doubly-nested
inductive, since the product (or pair) type itself is defined inductively, in
addition to the inductively defined polymorphic lists (\ref{lst:def-list}). In
this case, we need to manually define a stronger induction principle for use.
This can be done by noticing that nested inductive types have an equivalent
rewrite by expanding the definition explicitly. For example, in the case of rose
trees, Coq will generate the correct (sufficiently strong) induction principle
for the following equivalent definition:

\begin{listing}[H]
\begin{minted}{Coq}
Inductive roseTree :=
| node (xs: listRoseTree)
where listRoseTree :=
| nil
| cons (r: roseTree) (tl: listRoseTree).
\end{minted}
  \caption{Non-nested inductive definition of rose trees.}
  \label{lst:def-sf-nn}
\end{listing}

The reason why we prefer to use readily-available abstractions such as lists is
that one can then rely on facts that are proven already on the polymorphic
lists, and properties on their operations such as \verb|map| and \verb|fold|.
Therefore, the operation here is twofold:

\begin{enumerate}
  \item generate induction principle for the non-nested formulation and remember
  both the induction principle (a type in Coq, due to Curry-Howard
  correspondence) and the term that inhabits it.
  \item now redefine the type in a nested inductive manner, then manually define
  the induction principle and prove it. The corresponding action in Coq is to
  define the type corresponding to the inductive principle (which is similar to
  the one extracted above) while replacing the explicitly inductive parts with
  its relevant counterparts. For example, replace the explicit weakening
  described in (REFME) with the \verb|Coq.Lists.List.Forall| inductive
  (higher-order) predicate. Finally, prove this newly defined induction
  principle correct by supplying the term we obtained in the previous step with
  suitable amendments.
\end{enumerate}

We illustrate this with an actual implementation of the strengthened induction
principle on structure fields. First, we define the fixpoint \verb|F1| that
switch cases on the constructor of a typical list, \verb|nil| and \verb|cons|:

\begin{listing}[H]
\tcc{360}{367}{Environment.v}
  \caption{Manually proving IH for the nested list.}
  \label{lst:def-nn-ih}
\end{listing}

Then, we can integrate \verb|F1| into the induction principle terms for
structure fields (including its mutually inductive counterpart, module
implementation), therefore forming the proofs for the individual strengthened
induction principles:

\begin{listing}[H]
\tcc{369}{386}{Environment.v}
  \caption{Induction principles with explicit names.}
  \label{lst:def-nn-ih-plus}
\end{listing}

Finally, the last 3 lines give suggestive names to the fixpoints corresponding
to the induction hypothesis for the respective structures. Their types are
already inductive hypotheses about only one particular branch of the mutually
inductive definition, and we combine them into a huge induction principle:

\begin{listing}[H]
\tcc{389}{407}{Environment.v}
  \caption{The strengthened mutual induction principle.}
  \label{lst:def-nn-ihs}
\end{listing}

To define a type, one has to show that the type is inhabited, and the purpose of
the proof is exactly to construct such a term for the type. Till here, we have
seen a non-trivial example of how to define an induction principle for a nested
(and in addition, mutually) inductive type.

There has been some metaprogramming effort to generate nested induction
hypotheses in Coq automatically using the MetaCoq platform
(\cite{liesnikov2020generating}), however, the tool was not mature enough and
has not been updated since some previous versions.

\section{Well-founded recursion}

Recursion with a single recursive constructor and one (or more) base cases are
well-known, such as that of factorial or Fibonacci numbers. One can naturally
extend the definition to structural recursions, such as that on an abstract
syntax tree when writing programming language implementations. However, when the
recursive case is not a strict "predecessor" of the current case, we need a
stronger form of recursion. For the case of linearly ordered datatypes such as
natural numbers, we can argue using strong induction; however, for other
datatypes where there isn't a canonical linear order, such as, again, rose
trees (\ref{lst:def-rosetree}). 

Recursing on the items in the list of rose trees is essential, but to Coq they
are not direct subterms of the tree while the argument \verb|xs| is. We are left
with the difficult choice of convincing the positivity check for fixpoints in
Coq that our recursion terminates.

Fortunately, this can be solved by well-founded recursion, a more general form
of recursion that determines termination by the following:
\begin{enumerate}
  \item defining a linear order on the recursive term, called a \emph{measure}.
  \item showing that this measure is lower-bounded.
\end{enumerate}

Then the recursion is well-founded and will terminate. The use case of this in
the project is to recurse through a module and resolve the fully qualified name
(e.g. \verb|M.N.a|) from the nested module structure and their identifier names
(e.g. \verb|M, N, a|). For tree-like structures, a typical measure is the
\emph{height} of the tree:

\begin{listing}[H]
\tcc{415}{425}{Environment.v}
  \caption{Height defined on structure body.}
  \label{lst:def-sb-height}
\end{listing}

Then we show this measure is lower-bounded:

\begin{listing}[H]
\tcc{427}{430}{Environment.v}
  \caption{Proof of lower bound of the height measure.}
  \label{lst:def-height-lb}
\end{listing}

Therefore, we can define a well-founded recursion using this measure,
specifically mentioned on line 433: 

\begin{listing}[H]
\tcc{432}{434}{Environment.v}
...
\tcc{446}{448}{Environment.v}
\tcc{466}{469}{Environment.v}
  \caption{Defining a well-founded recursive function using the new measure.}
  \label{lst:def-wf}
\end{listing}


\section{Strengthening the induction hypothesis}
\label{sec:str-ih}

Sometimes, the induction hypothesis of a statement is too weak and entirely
useless. For example, suppose we wish to show the following:

\begin{theorem}
  Fix a natural number $x$. Then for all natural numbers $n\ge1$, $x+n\ne x$.
\end{theorem}
\begin{proof}[Wrong proof.]
Let us attempt a direct induction.

Base case: suppose $n=1$, $x+n=n+1=succ(x)\ne x$ by definition.

Inductive case: suppose any $n\ge 1$, $x+n\ne x$. Then
\begin{align*}
  x + (n+1) = (x + n) + 1 = succ(x + n)
\end{align*}

Now, we wish to say
\begin{align*}
  x + n \ne x \implies succ(x + n) \ne x
\end{align*}
which is not true in general!
\end{proof}

The key here is that the induction hypothesis of "not equal" is too weak and
does not help with proving the inductive case. Should it have been the strict
inequality instead, $x+ n>x$ which implies $x+n\ne x$, we would have been fine!

Put in context, we wanted to show that for every declaration with kername
\verb|kn'| in a module named \verb|kn|, can not have the same name as the module
after translation.

\begin{listing}[H]
\pcuic{502}{503}{TemplateToPCUIC.v}
  \caption{A module cannot have the same kername as its contents.}
  \label{lst:def-kn-neq}
\end{listing}

The direct induction proved to be a failure. However, we can prove instead a
stronger statement:

\begin{listing}[H]
\pcuic{350}{352}{TemplateToPCUIC.v}
  \caption{Every declaration within a module must have a kername extending that of the module.}
  \label{lst:def-kn-ext}
\end{listing}

Since the binary relation on kernames, \verb|kn_extends kn kn'| decides whether
\verb|kn'| is an extension of \verb|kn| and is a strict partial order, it would
imply the inequality in \ref{lst:def-kn-neq}. Then, it remains to define it in
Coq and show the asymmetry and transitivity of this relation. We omit the proof
here.

\begin{listing}[H]
\pcuic{152}{154}{TemplateToPCUIC.v}
\pcuic{190}{191}{TemplateToPCUIC.v}
\pcuic{239}{239}{TemplateToPCUIC.v}
  \caption{Definition, asymmetry and transitivity of the relation.}
  \label{lst:kn-ext-po}
\end{listing}