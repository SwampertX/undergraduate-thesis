\section{Implementation}

\newcommand{\tc}[3]{\inputminted[firstline={#1},lastline={#2},linenos]{Coq}{
code/v1/template-coq/theories/#3}}
\newcommand{\pcuic}[3]{\inputminted[firstline={#1},lastline={#2},linenos]{Coq}{
code/v1/pcuic/theories/#3}}

\begin{frame}
  \frametitle{Outline}
  \tableofcontents[currentsection]
\end{frame}

\begin{frame}
  \frametitle{Roadmap}
  \begin{center}
  \begin{tabular}{ |l|l| }
    \hline
    Implementation & Verification \\
    \hline\hline
    {1. Definition of Modules }& 2. Lookup of definitions\\
    3. Typing rules for Modules & 4. Functoriality of Typing Rules\\
    \emph{(Term typing rules)} & 5. Typing of terms\\
    6. Translation to PCUIC & (Correctness of translation)\\
    \hline
    7. Modular Environment & (Correctness of implementation)\\
    \hline\hline
  \end{tabular}
  \end{center}
  8. Three Formal Proof Techniques
\end{frame}

\subsection{First Implementation}

\begin{frame}
  \frametitle{1. Definition of Modules}
  Definition of Structures.
\begin{listing}[H]
  \tc{324}{336}{Environment.v}
  \caption{TemplateCoq/theories/Environment.v}
\end{listing}
\end{frame}

\begin{frame}
  \frametitle{1. Definition of Modules}
Now, we can define proper Modules and Module Types as follows:
\begin{listing}[H]
  \tc{344}{345}{Environment.v}
  \tc{347}{351}{Environment.v}
  \caption{TemplateCoq/theories/Environment.v}
\end{listing}
\end{frame}

\begin{frame}
  \frametitle{2. Lookup of Modules}
  \begin{theorem}[Lookup]
  Looking up \texttt{kn} yields \texttt{mdecl} iff \texttt{mdecl} is declared
  with \texttt{kn}.
  \end{theorem}
\begin{listing}[H]
  \tc{202}{215}{EnvironmentTyping.v}
  \caption{TemplateCoq/theories/EnvironmentTyping.v}
\end{listing}
\end{frame}
  
\begin{frame}
  \frametitle{3. Typing rules for modules}
  The core is the structure fields.
  \begin{listing}[H]
    \tc{1223}{1232}{EnvironmentTyping.v}
    \caption{Typing rules for structure fields.}
\end{listing}
\end{frame}

\begin{frame}
  \frametitle{3. Typing rules for modules}
  Subsequently, the typing rule for structures, and modules.
  \begin{listing}[H]
    \tc{1233}{1243}{EnvironmentTyping.v}
    \tc{1250}{1252}{EnvironmentTyping.v}
    \caption{Typing rules for structure, and modules.}
\end{listing}
\end{frame}

\begin{frame}
  \frametitle{4. Functoriality of Typing Rules}
  \begin{lemma}[Global declaration]
    Fix term typing rules $P,Q$ such that if the environment is $P$-well-formed
    if $P$ types term $t$ with type $T$, then $Q$ types term $t$ with type $T$
    as well.
    
    Let $\Sigma$ be a $P$-well-formed environment. If the definition $(kn,d)$ is
    well-formed, then $(kn,d)$ is $Q$-well-formed.
  \end{lemma}
  \begin{listing}[H]
    \tc{1431}{1436}{EnvironmentTyping.v}
    \caption{Functoriality of typing of a global declaration.}
  \end{listing}
\end{frame}

\begin{frame}
  \frametitle{4. Functoriality of Typing Rules}
  \begin{theorem}[Global Environment]
    Fix term typing rules $P,Q$ such that they type terms in the same way for
    all terms $t:T$.
    
    Let $\Sigma$ be a $P$-well-formed environment. Then $\Sigma$ is
    $Q$-well-formed.
  \end{theorem}
  \begin{listing}[H]
    \tc{1459}{1464}{EnvironmentTyping.v}
    \caption{Functoriality of the typing of global environments.}
  \end{listing}
\end{frame}

\begin{frame}
  \frametitle{5. Typing of terms}
  \begin{theorem}
    Fix any two predicates $P$ and $P_\Gamma$ that about a term $t$ and
    a type $T$. Suppose we are given global environment $\Sigma$ and local
    context $\Gamma$ which are well-formed, and that the following typing
    relation holds: $\Sigma ;; \Gamma \vdash t:T$, then $P$ holds on the
    global environment $\Sigma$, and $P_\Gamma$ holds on the local context. 
  \end{theorem}
  \begin{listing}[H]
    \tc{1020}{1025}{Typing.v}
    \caption{Definition of key lemma in typing.}
  \end{listing} 
\end{frame}

\begin{frame}
  \frametitle{Checkpoint 1!}

  This marks the end of the TemplateCoq part of the First Implementation.
  We have seen
  \begin{enumerate}
  \item The definition of Modules.
  \item Proof of lookup iff declared.
  \item The definition of Typing Rules.
  \item Functoriality.
  \item Typing properties of terms.
  \end{enumerate}

  We will show the translation to PCUIC and motivate the Second Implementation.

\end{frame}

\begin{frame}
  \frametitle{6. Translation to PCUIC}
  The global environment for PCUIC is without modules:

  \begin{listing}[H]
    \pcuic{278}{288}{Environment.v}
    \caption{Definition of the global environment for PCUIC.}
  \end{listing}
  
  So we translate by ... removing modules!
\end{frame}

\begin{frame}
  \frametitle{6. Translation to PCUIC}
  The engine of the translation of modules.
  \begin{listing}[H]
    \pcuic{314}{325}{TemplateToPCUIC.v}
    \caption{Translation of structure fields to PCUIC.}
  \end{listing}
\end{frame}

\begin{frame}
  \frametitle{6. Translation to PCUIC}
  Run the field-by-field translation over the body.
  \begin{listing}[H]
    \pcuic{334}{339}{TemplateToPCUIC.v}
      \caption{Translating structure body.}
    \end{listing}
\end{frame}

\begin{frame}
  \frametitle{6. Translation to PCUIC}
  Now we can translate a global declaration...
  \begin{listing}[H]
    \pcuic{508}{516}{TemplateToPCUIC.v}
      \caption{Translating a global declaration.}
    \end{listing}
\end{frame}

\begin{frame}
  \frametitle{6. Translation to PCUIC}
  And finally global declarations!
  \begin{listing}[H]
    \pcuic{527}{531}{TemplateToPCUIC.v}
      \caption{Translating global declarations.}
    \end{listing}
  \pause
  Uh-oh... notice the double fold.
\end{frame}

\begin{frame}
  \frametitle{6.5. Verification of translation}
  \begin{theorem}[Translated iff Exists]
    "Translation preserves non-existence", that is, the translated environment
    should only contain the intended translation and nothing more; and its dual,
    "Translation preserves existence", that is, nothing is lost in translation.
  \end{theorem}
\end{frame}

\begin{frame}
  \frametitle{6.5. Verification of translation}

  

\end{frame}

\begin{frame}
  \frametitle{6.9. Motivation for Second Implementation}
  \begin{listing}[H]
    \pcuic{239}{241}{TemplateToPCUICCorrectness.v}
    \pcuic{307}{308}{TemplateToPCUICCorrectness.v}
    \pcuic{316}{318}{TemplateToPCUICCorrectness.v}
    \pcuic{326}{326}{TemplateToPCUICCorrectness.v}
    ...
      \caption{Tedious nested proofs.}
    \end{listing}
    The first case takes 200 lines and counting! \pause
    Too many repeated proofs.
\end{frame}


\begin{frame}
  \frametitle{6.9. Motivation for Second Implementation}
  Culprit!
\begin{listing}[H]
  \tc{324}{328}{Environment.v}
  \tc{347}{351}{Environment.v}
  \caption{An opportunity for abstraction!}
\end{listing}
\end{frame}

\subsection{Second Implementation - Modular Environment}

\begin{frame}
  \frametitle{7. The Modular Environment}
  As the name suggests, we combine the concepts of structures (modules) and
  environments.\pause

  An environment is just a module \pause named by its directory path (eg.
  \texttt{~/metacoq/template-coq/theories/Environment.v}).\pause

  All theorems on the typing of environment follow from that of modules!
\end{frame}

\newcommand{\tcc}[3]{\inputminted[firstline={#1},lastline={#2},linenos]{Coq}{
  code/v2/template-coq/theories/#3}}
\newcommand{\pcuicc}[3]{\inputminted[firstline={#1},lastline={#2},linenos]{Coq}{
  code/v2/pcuic/theories/#3}}

\begin{frame}
  \frametitle{7. The Modular Environment}
  Let us define modules, then specialize into environments.
  \begin{listing}[H]
    \tcc{325}{332}{Environment.v}
      \caption{Definition of structure fields.}
    \end{listing}
\end{frame}

\begin{frame}
  \frametitle{7. The Modular Environment}
  ``Globalization''!
  \begin{listing}[H]
    \tcc{409}{412}{Environment.v}
    \caption{Definition of global declarations.}
  \end{listing}
\end{frame}

\begin{frame}
  \frametitle{7.5. Typing Rules}
  Implemented but unverified typing rules. The interesting part follows...
  \begin{listing}[H]
    \tcc{1271}{1282}{EnvironmentTyping.v}
    \caption{Typing rules for structure fields.}
  \end{listing}
\end{frame}

\begin{frame}
  \frametitle{7.5. Typing Rules}
  Now structure bodies encompass the typing of environments, such as the
  freshness of names.
  \begin{listing}[H]
    \tcc{1284}{1291}{EnvironmentTyping.v}
      \caption{Typing rules of structure body.}
    \end{listing}
\end{frame}

\subsection{Formal Proof Techniques}

\begin{frame}
  \frametitle{Recursion, recursion, recursion}
  
  All three techniques are related to recursion and were investigated during
  the modular environment rewrite.
  \begin{enumerate}
  \item Stronger Induction Principle for Nested Inductive Types
  \item Well-formed Recursion
  \item Strengthening of Induction Hypothesis (omitted)
  \end{enumerate}
\end{frame}

\begin{frame}
  \frametitle{8.1. Nested Inductive Types}
  Inductive type within an inductive type.

  Rose tree (Meertens 1998):

  \begin{listing}[H]
    \inputminted[firstline=1,lastline=2]{Coq}{code/machineries.v}
  \caption{Definition of a rose tree.}
  \end{listing}
\end{frame}

\begin{frame}
  \frametitle{8.1. Nested Inductive Types}

  Unfortunately, Coq does not generate a strong enough induction principle
  for nested inductive types, only the below:

\begin{align*}
  \forall P, (\forall xs, P (node\ xs)) \implies \forall rt, (P\ rt)
\end{align*}

We need to check each rose tree within the list with predicate $P$ first. \pause
Here is a stronger induction principle that is generally used:

\begin{align*}
  \forall P, (\forall xs, (\forall x\in xs, P\ x) \implies P (node\ xs))
   \implies \forall rt, (P\ rt)
\end{align*}

The induction hypothesis is weakened, and the induction principle is
strengthened!

\end{frame}

\begin{frame}
  \frametitle{8.1. Where is this used?}
  In the modular rewrite - definition of structures!
  \begin{listing}[H]
    \tcc{325}{332}{Environment.v}
      \caption{Definition of structure fields.}
    \end{listing}
\end{frame}

\begin{frame}
  \frametitle{8.2. Well-founded recursion}
  Typical recursion: predecessor. What if this is not obvious?\pause
  
  Define a measure, and show that it is
  \begin{itemize}
    \item bounded below, and
    \item strictly decreasing at every recursive step.
  \end{itemize}\pause
\end{frame}

\begin{frame}
  \frametitle{8.2. Where is this used?}
  To recurse through the nested inductive structure body! Here is a measure:
  \begin{listing}[H]
    \tcc{415}{425}{Environment.v}
      \caption{Height defined on structure body.}
    \end{listing}
\end{frame}

\begin{frame}
  \frametitle{8.2. Where is this used?}
  \begin{listing}[H]
    \tcc{427}{430}{Environment.v}
      \caption{Proof of lower bound of the height measure.}
      \label{lst:def-height-lb}
    \end{listing}
\end{frame}