\section{Introduction}
\subsection{Summary}

\begin{frame}
  \frametitle{Motivation}
  % \begin{itemize}
  %   \item Coq is a proof assistant for formal verification.
  %   \item Widely used for mathematical proofs and program verification.
  %   \item Implementation of Coq has consistency-threatening bugs!
  % \end{itemize}
  Coq is a proof assistant for formal verification in Intuitionistic
  (Constructive) Logic.\pause
  
  Widely used for mathematical proofs (such as the Four-Color Theorem and
  Feit-Thompson Theorem) and program verification (CompCert C Compiler).
  \pause

  Implementation of Coq has consistency-threatening bugs! Who watches the
  watchers?\pause

  Or can Coq verify itself?
\end{frame}

\begin{frame}
  \frametitle{The MetaCoq Project}
  A metaprogramming platform for Coq (TemplateCoq), turned into a verified
  implementation of Coq in Coq.\pause
  
  In 2020, Sozeau et. al. completed the formalization for a large subset of Coq
  in the MetaCoq project: "Coq Coq Correct! Verification of Type Checking and
  Erasure for Coq, in Coq".\pause

  However, a few features such as Modules are missing from the project. Modules
  are important for almost all large Coq projects!\pause

  Therefore we are here!
\end{frame}

\begin{frame}
  \frametitle{Contributions of This Project}
  \begin{enumerate}
    \item A Coq implementation of non-parametrized Coq modules within the
    MetaCoq framework, at the TemplateCoq level.\pause
    \item Verification of correctness properties related to modules,
    environment, and typing.\pause
    \item Translation of modules from TemplateCoq to PCUIC.\pause
    \item A second implementation of Coq modules unifying Modules and the Global
    Environment.\pause
    \item A summary of three recursion-related formal proof techniques.\pause
  \end{enumerate}
\end{frame}