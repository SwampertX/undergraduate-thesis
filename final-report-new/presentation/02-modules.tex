% \section{Background}

\subsection{The MetaCoq Project}

\begin{frame}
  \frametitle{The MetaCoq Project - History}
  A metaprogramming platform for Coq.

  Originally \textbf{TemplateCoq}, a Coq program that reifies/quotes terms in
  Coq.\pause

  Since we have terms, why not state some properties about them?\pause
    
  Added \textbf{PCUIC} (Polymorphic Cumulative Calculus of Inductive
  Constructions) and \textbf{Safechecker}, a fuel-free, verified reduction
  machine.
  \pause

  \textbf{Proof erasure} to untyped calculus, ready for translation into "usual"
  programming languages. \pause
\end{frame}

\begin{frame}
  \frametitle{Where is the implementation?}
  \begin{itemize}
    \item (Coq) -- TemplateCoq -- PCUIC -- Checker -- Erasure -- (Machine Code)
    \item Actual data structure of modules live in \textbf{TemplateCoq}.
    \item Verification of properties of modules live in \textbf{TemplateCoq}.
    \item Translation from \textbf{TemplateCoq} to \textbf{PCUIC}.
    \item Difference? PCUIC is easier to prove (semantical) theorems.
  \end{itemize}
\end{frame}

\subsection{Syntax and Semantics of Coq Modules}
\begin{frame}
  \frametitle{Example - Definitions}
  Modules as ``collections of definitions''.
  \inputminted[firstline=1,lastline=10]{coq}{code/module_example.v}
\end{frame}

\begin{frame}
  \frametitle{Example - Modules}
  ``Packaging'' definitions into a Module (Type).
  \inputminted[firstline=11,lastline=16]{coq}{code/module_example.v}
  \inputminted[firstline=19,lastline=23]{coq}{code/module_example.v}
\end{frame}

\begin{frame}
  \frametitle{Example - Aliasing}
  Modules can be aliased for ease of reference.
  \inputminted[firstline=24,lastline=25]{coq}{code/module_example.v}
\end{frame}

\begin{frame}
  \frametitle{Example - Functors}
  Higher-order modules - Functors.
  \inputminted[firstline=27]{coq}{code/module_example.v}
\end{frame}

\begin{frame}
  \frametitle{Abstract Syntax of Coq Modules}
  A \textbf{structure} is an ordered list of declarations of the following
  kinds:
  \begin{itemize}
    \item A \textbf{constant} declaration.
    \item An \textbf{inductive} declaration.
    \item A \textbf{module} declaration.
    \item A \textbf{module type} declaration.
  \end{itemize}\pause
  A \textbf{module} is a \textbf{structure} with a name and possibly a
  \textbf{module type}, where all definitions are concrete.\pause
  
  A \textbf{module type} is a \textbf{structure} with a name.\pause

  A \textbf{functor} is a parametrized module, by another module or functor.
\end{frame}

\begin{frame}
  \frametitle{Semantics of Coq Modules}
  Modules are declarations, and they live in an \textbf{environment}. An
  environment is an ordered list of declarations:
  \begin{itemize}
    \item A \textbf{constant} declaration.
    \item An \textbf{inductive} declaration.
    \item A \textbf{module} declaration.
    \item A \textbf{module type} declaration.
  \end{itemize}
\end{frame}

\newcommand{\wf}[1]{\mathrm{WF}(#1)}
\newcommand{\struct}[1]{\text{Struct }#1\text{ End}}
\newcommand{\module}[1]{\text{Mod}(#1)}
\newcommand{\moda}[1]{\text{ModA}(#1)}
\newcommand{\modtype}[1]{\text{ModType}(#1)}
\newcommand{\with}{\text{ with }}
\newcommand{\range} % range(variable, start, end)
[3]{{#1}_{#2};\ldots;{#1}_{#3}}
\begin{frame}
  \frametitle{Semantics of Coq Modules}
  Coq Modules are second-class objects and have separate semantics from that of
  terms. Lives on another plane and have limited interactions.

  Semantics are given by typing rules. Formation rules and access rules.

  \[
  \inferrule{\wf{E,E'}[]}{E[]\vdash\wf{\struct{E'}}}\quad
  \]
  \[
  \inferrule{
      E[]\vdash p\to\struct{\range{e}{1}{i};\module{X:S[:=S_1]};\range{e}{i+2}{n}}\\\\    
      E;\range{e}{1}{i}[]\vdash S\to\overline{S}
  }{
    E[]\vdash p.X\to\overline{S}
  }
  \]
\end{frame}