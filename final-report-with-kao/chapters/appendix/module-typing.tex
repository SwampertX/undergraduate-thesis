\chapter{Typing Rules for Coq Modules}
\label{chap:typing}
% \section{Specification of Coq Modules}

\newcommand{\wf}[1]{\mathrm{WF}(#1)}
\newcommand{\struct}[1]{\text{Struct }#1\text{ End}}
\newcommand{\module}[1]{\text{Mod}(#1)}
\newcommand{\moda}[1]{\text{ModA}(#1)}
\newcommand{\modtype}[1]{\text{ModType}(#1)}
\newcommand{\with}{\text{ with }}
% \newcommand{\range} %range([separator=;], variable, start, end)
% [4][;]{{#2}_{#3}{#1}\ldots{#1}{#2}_{#4}}
\newcommand{\range} % range(variable, start, end)
[3]{{#1}_{#2};\ldots;{#1}_{#3}}

Typing rules here are given by
\href{https://coq.inria.fr/refman/language/core/modules.html#typing-modules}{Coq: Typing Modules}.

\section{Overview}
\label{sec:typing-overview}
This appendix is divided into 5 parts. Section \ref{sec:typing-overview} will
first define the necessary terminology. Then, sections \ref{sec:typing-formation}
to \ref{sec:typing-typing} will define the typing rules of Coq Modules. They
correspond to the 4 operations defined on modules:

\begin{enumerate}
\item \textbf{Formation Rules} (\ref{sec:typing-formation}) describes how to form
modules.
\item \textbf{Evaluation Rules} (\ref{sec:typing-evaluation}) describes how to
evaluate specific terms involving modules: with-expressions and path-expressions.
\item \textbf{Access Rules} (\ref{sec:typing-access}) describes how to access
contents of modules (i.e. definitions in modules) using pathnames.
\item \textbf{Typing Rules} (\ref{sec:typing-typing}) describes how a module is 
inductively typed.
\end{enumerate}

\subsection{Definitions}
\label{sec:typing-overview-definitions}
A structure is ...
A module type is ...
FIXME:
\subsection{Judgement Rules}
\label{sec:typing-overview-judgement}
\subsection{Variables}
\label{sec:typing-overview-variables}
E represents environment, S represents structures, ...

\section{Formation Rules}
\label{sec:typing-formation}

WF-Struct: 
\marginnote{A structure can be formed from the tail of a well-formed environment.}
\[
\inferrule{\wf{E,E'}[]}{E[]\vdash\wf{\struct{E'}}}
\]

WF-Mod1
\marginnote{If structure $S$ is well-formed under the well-formed environment $E$, then
the environment containing the Module $X$ of type $S$ is well-formed.}
\[
\inferrule{
    \wf{E}[]\\ E[]\vdash\wf{S}
}{
    \wf{E;\module{X:S}}[]
}    
\]

WF-Mod2
\marginnote{The previous also works if $X$ is of type $S_2$, a subtype of $S_1$.}
\[
\inferrule{
    E[]\vdash S_2<:S_1\\
    \wf{E}[]\\
    E[]\vdash\wf{S_1}\\
    E[]\vdash\wf{S_2}
}{
    \wf{E;\module{X:S_1:=S_2}}[]
}    
\]

WF-Alias
\marginnote{If the module pointed by $p$ has type $S$ in the global environment $E$,
then we can create an aliased module $X$ to $p$. This implies that $p\in E$. }
\[
\inferrule{
    \wf{E}[]\\
    E[]\vdash p:S
}{
    \wf{E;\moda{X==p}}[]
}    
\]

WF-ModType
\marginnote{A ModType is a named well-formed structure.}
\[
\inferrule{
    \wf{E}[]\\
    E[]\vdash\wf{S}
}{
    \wf{E;\modtype{Y:=S}}[]
}    
\]

WF-Ind
\marginnote{An inductive definition strenghtened by $p$ is well-formed if 
it is already well-formed in $E$ and it is within $p$. This can be generalized
with subtyping.}
\[
\inferrule{
    \wf{E;\text{Ind}[r](\Gamma_I:=\Gamma_C)}[]\\
    E[]\vdash p:\struct{\range{e}{1}{n};\text{Ind}[r](\Gamma^{'}_I:=\Gamma^{'}_C);\ldots}\\
    E[]\vdash \text{Ind}[r](\Gamma^{'}_I:=\Gamma^{'}_C)<: \text{Ind}[r](\Gamma_I:=\Gamma_C)\\
}{
    \wf{E;\text{Ind}_p[r](\Gamma_I:=\Gamma_C)}[]
}    
\]

\section{Evaluation Rules}
\label{sec:typing-evaluation}
\subsection{Evaluating \texttt{with} expressions}
WEval-With-Mod:
\[
\inferrule
{E[]\vdash S\to\struct{e_1;\ldots;e_i;\module{X:S_1};e_{i+2};\ldots;e_n}\\
    E;e_1;\ldots;e_i[]\vdash S_1\to \overline{S_1}\\
    E[]\vdash p:S_2\\
    E;e_1;\ldots;e_i[]\vdash S_2<:\overline{S_1}}
{E[]\vdash S\with X:=p\to\\\\
    \struct{\range{e}{1}{i};\moda{X==p};e_{i+2}\{X/p\};\ldots;e_n\{X/p\}}}
\]


WEval-With-Mod-Rec:
\[
\inferrule
{E[]\vdash S\to\struct{e_1;\ldots;e_i;\module{X:S_1};e_{i+2};\ldots;e_n}\\
    E;\range{e}{1}{i}[]\vdash S_1\with p:=p_1\to\overline{S_2}}
{E[]\vdash S\with X_1.p:=p_1\to\\\\
    \struct{\range{e}{1}{i};\mod{X:\overline{S_2}};e_{i+2}\{X.p/p_1\};\ldots;e_n\{X.p/p_1\}}}
\]

WEval-With-Def
\marginnote{This is the base case.}
\[
\inferrule{
    E[]\vdash S\to\struct{\range{e}{1}{i};(c:T_1);\range{e}{i+2}{n}}\\
    E;\range{e}{1}{i}[]\vdash(c:=t:T)<:(c:T_1)
}{
    E[]\vdash S\with c:=t:T\to
    \struct{\range{e}{1}{i};(c:=t:T);\range{e}{i+2}{n}}
}  
\]


WEval-With-Def-Rec
\[
\inferrule{
    E[]\vdash S\to\struct{\range{e}{1}{i};\module{X_1:S_1};\range{e}{i+2}{n}}\\
    E;\range{e}{1}{i}[]\vdash S_1\with p:=p_1\to\overline{S_2}
}{
    E[]\vdash S\with X_1.p:=t:T\to\\\\
    \struct{\range{e}{1}{i};\module{X:\overline{S_2}};\range{e}{i+2}{n}}
}  
\]

\subsection{Evaluating paths ($p.X$)}
WEval-Path-Mod1
\[
\inferrule{
    E[]\vdash p\to\struct{\range{e}{1}{i};\module{X:S[:=S_1]};\range{e}{i+2}{n}}\\\\    
    E;\range{e}{1}{i}[]\vdash S\to\overline{S}
}{
    E[]\vdash p.X\to\overline{S}
}
\]

WEval-Path-Mod2
\[
\inferrule{
    \wf{E}[]\\
    \module{X:S[:=S_1]}\in E\\
    E[]\vdash S\to\overline{S}
}{
    E[]\vdash X\to\overline{S}
}
\]

WEval-Path-Alias1
\[
\inferrule{
    E[]\vdash p\to\struct{\range{e}{1}{i};\moda{X==p_1};\range{e}{i+2}{n}}\\\\    
    E;\range{e}{1}{i}[]\vdash p_1\to\overline{S}
}{
    E[]\vdash p.X\to\overline{S}
}
\]

WEval-Path-Alias2
\[
\inferrule{
    \wf{E}[]\\
    \moda{X==p_1}\in E\\
    E[]\vdash p_1\to\overline{S}
}{
    E[]\vdash X\to\overline{S}
}
\]

WEval-Path-Type1
\[
\inferrule{
    E[]\vdash p\to\struct{\range{e}{1}{i};\modtype{Y:=S};\range{e}{i+2}{n}}\\\\    
    E;\range{e}{1}{i}[]\vdash S\to\overline{S}
}{
    E[]\vdash p.Y\to\overline{S}
}
\]

WEval-Path-Type2
\[
\inferrule{
    \wf{E}[]\\
    \modtype{Y:=S}\in E\\
    E[]\vdash S\to\overline{S}
}{
    E[]\vdash Y\to\overline{S}
}
\]

\section{Access Rules}
\label{sec:typing-access}


\section{Typing Rules}
\label{sec:typing-typing}
MT-Eval
\[
\inferrule{
    E[]\vdash p\to\overline{S}
}{
    E[]\vdash p:\overline{S}
}      
\]

MT-Str
\[
\inferrule{
    E[]\vdash p\to\overline{S}
}{
    E[]\vdash p:S/p
}      
\]

Where $S/p$ is the strengthening operation, defined on $S$ where $S\to\struct{\range{e}{1}{n}}$ as:
\begin{align*}
(c:=t:T)/p &= (c:=t:T)\\
(c:U)/p &= (c:=p.c:U)\\
\module{X:S}/p &= \moda{X==p.X}\\
\moda{X==p'}/p &= \moda{X==p'}\\
\text{Ind}[r](\Gamma_I :=\Gamma_C)/p &= \text{Ind}_p[r](\Gamma_I :=\Gamma_C)\\
\text{Ind}_{p'}[r](\Gamma_I :=\Gamma_C)/p &= \text{Ind}_{p'}[r](\Gamma_I :=\Gamma_C)
\end{align*}
It strengthens the definition of inductives to equal to the current path only,
solving the issue of equality of defined inductive types in modules.

\subsection{Subtyping rules}
MSub-Str 
\[
\inferrule{
    E;\range{e}{1}{n}[]\vdash e_{\sigma(i)}<: e'_i \text{ for }i=1,\ldots,m\\\\
    \sigma:\{1,\ldots,m\}\to\{1,\ldots,n\} \text{ injective}
}{
 E[]\vdash\struct{\range{e}{1}{n}}<:\struct{\range{e'}{1}{m}}
}    
\]

Mod-Mod:
\[
\inferrule{
    E[]\vdash S_1<:S_2
}{
    E[]\vdash\module{X:S_1}<:\module{X:S_2}
}    
\]

Alias-Mod:
\[
\inferrule{
    E[]\vdash p:S_1
    E[]\vdash S_1<:S_2
}{
    E[]\vdash\moda{X==p}<:\module{X:S_2}
}    
\]

Mod-Alias:
\[
\inferrule{
    E[]\vdash p:S_2
    E[]\vdash S_1<:S_2
    E[]\vdash X=_{\beta\delta\iota\zeta\eta}p
}{
    E[]\vdash\module{X:S_1}<:\moda{X==p}
}    
\]

Alias-Alias
\[
\inferrule{
    E[]\vdash p_1=_{\beta\delta\iota\zeta\eta}p_2
}{
    E[]\vdash\moda{X==p_1}<:\moda{X==p_2}
}    
\]

Modtype-Modtype
\[
\inferrule{
    E[]\vdash S_1<:S_2
    E[]\vdash S_2<:S_1
}{
    E[]\vdash\modtype{Y:=S_1}<:\modtype{Y:=S_2}
}    
\]


\subsubsection{Structure element subtyping rules}
Assum-Assum
\[
\inferrule{
    E[]\vdash T_1\le_{\beta\delta\iota\zeta\eta}T_2
}{
    E[]\vdash(c:T_1)<:(c:T_2)
}    
\]


Def-Assum
\[
\inferrule{
    E[]\vdash T_1\le_{\beta\delta\iota\zeta\eta}T_2
}{
    E[]\vdash(c:=t:T_1)<:(c:T_2)
}    
\]

Assum-Def
\[
\inferrule{
    E[]\vdash T_1\le_{\beta\delta\iota\zeta\eta}T_2\\
    E[]\vdash c=_{\beta\delta\iota\zeta\eta}t_2
}{
    E[]\vdash(c:T_1)<:(c:=t_2:T_2)
}    
\]

Def-Def
\[
\inferrule{
    E[]\vdash T_1\le_{\beta\delta\iota\zeta\eta}T_2\\
    E[]\vdash t_1=_{\beta\delta\iota\zeta\eta}t_2
}{
    E[]\vdash(c:=t_1:T_1)<:(c:=t_2:T_2)
}    
\]

Ind-Ind
\[
\inferrule{
    E[]\vdash \Gamma_I=_{\beta\delta\iota\zeta\eta}\Gamma'_I\\
    E[\Gamma_I]\vdash \Gamma_C=_{\beta\delta\iota\zeta\eta}\Gamma'_C
}{
    \text{Ind}[r](\Gamma_I :=\Gamma_C) <: \text{Ind}[r](\Gamma'_I :=\Gamma'_C)
}    
\]

Indp-Ind
\[
\inferrule{
    E[]\vdash \Gamma_I=_{\beta\delta\iota\zeta\eta}\Gamma'_I\\
    E[\Gamma_I]\vdash \Gamma_C=_{\beta\delta\iota\zeta\eta}\Gamma'_C
}{
    \text{Ind}_p[r](\Gamma_I :=\Gamma_C) <: \text{Ind}[r](\Gamma'_I :=\Gamma'_C)
}    
\]

Indp-Indp
\[
\inferrule{
    E[]\vdash \Gamma_I=_{\beta\delta\iota\zeta\eta}\Gamma'_I\\
    E[\Gamma_I]\vdash \Gamma_C=_{\beta\delta\iota\zeta\eta}\Gamma'_C\\
    E[]\vdash p={\beta\delta\iota\zeta\eta}p'
}{
    \text{Ind}_p[r](\Gamma_I :=\Gamma_C) <: \text{Ind}_{p'}[r](\Gamma'_I :=\Gamma'_C)
}    
\]