\chapter{Types and Proof Assistants}
% \section{Types and Proof Assistants}

\section{What are proof assistants?}
% \subsection{What are proof assistants?}
Proof assistants are a special kind of computer program whose job is to verify
the correctness of a mathematical proof. Informally, it has to verify if a
conclusion can be made by applying a fixed set of logical deduction steps on a
set of fixed assumptions. There are several ways to implement this; modern
implementations of proof assistants usually utilizes theories such as Higher
Order Logic (e.g. Isabelle, HOL Light), or some form of Type Theory.

In order to have a computer program that can do such powerful logical
reasoning to contain sophisticated mathematical results, the first 
question that arises is how to represent a theorem and what qualifies
as a valid proof of a theorem. Fortunately, Type Theory gives a simple
and direct relationship between Logic and Types.

\section{Type Theory}
% \subsection{Type Theory}

Types were first invented by Bertrand Russell as a way to provide higher-order
structures than sets to overcome his famous Russell's paradox. However, it was
Alonzo Church's Lambda Calculus which when used with types (Simply Typed Lambda
Calculus, or STLC) exhibited many desirable properties. STLC turned out to be a
consistent formal system, and a rewriting system that exhibits strong
normalization, which means the simplification of STLC terms will always
terminate, and always tends towards a unique normal form.

Most importantly, type systems such as STLC exhibits the Curry-Howard Correspondence,
which is the crucial observation that allows Type Theories to be useful for

\section{The Curry-Howard Correspondence}
% \subsection{The Curry-Howard Correspondence}

The Curry-Howard Correspondence establishes a one-to-one correspondence between
two distinct disciplines --- Logic and Types. It asserts that there is a
one-to-one correspondence between Types and Propositions, and between Terms and
Proofs. More precisely, it states that every proposition in intuitionistic
propositional logic (constructive 0-th order logic) can be expressed as a Type
in Simply Typed Lambda Calculus, and correspondingly, a term of a specific type
is viewed as a proof for the proposition represented by the type.

\begin{tabular}{c|c}
Intuitionistic Propositional Logic & Type Theory\\
\hline
Proposition & Type\\
Proof & Term
\end{tabular}

What are those exactly?

\subsection{Intuitionistic Propositional Logic}
% \subsubsection{Intuitionistic Propositional Logic}
There are two parts: Intuitionistic, and Propositional Logic. Let us start with
Propositional Logic.

\subsubsection{Propositional Logic} Propositional logic, (sometimes \emph{0-th
order logic}) is a logical system where propositions are constructed with
variables and logical connectors, such as $\to$ (implication), $\vee$
(disjunction), $\neg$ (negation), $\wedge$ (conjunction) and more, but not
including quantification ($\exists$ (exists), $\forall$ (for all)). For example,
propositions in Propositional Logic might look like the following:

\begin{enumerate}
    \item $p\vee q$
    \item $p\wedge \neg p$
    \item $(p\to q)\to (\neg q\to\neg p)$
\end{enumerate}

In \textbf{Classical} Propositional Logic, one can assign a valuation to each variable,
either $1$ or $0$, which represents "True" or "False" informally, and extend the
definition of valuation to propositions in a ``natural'' fashion: 

\begin{align*}
\tilde{v}(p)&=v(p)&&\text{if $p$ is a variable}\\
\tilde{v}(p\wedge q)&=1&&\text{iff }\tilde{v}(p) = 1 \text{ and } \tilde{v}(q) = 1\\
\tilde{v}(p\vee q)&=1&&\text{iff }\tilde{v}(p) = 1 \text{ or } \tilde{v}(q) = 1
\end{align*}

And so on. If there is some valuation that result in the proposition result in
$1$, then we say the proposition is \emph{satisfiable}; if there is none, i.e.
all valuations will result in the proposition being $0$, we say the proposition
is a \emph{contradiction}; on the other hand, if any valuation will result in
$1$, the proposition is a \emph{tautology}.

In particular, in the above example, 

\begin{enumerate}
    \item $p\vee q$ is satisfiable by $v(p)=1, v(q)=0$,
    \item $p\wedge \neg p$ is a contradiction, and 
    \item $(p\to q)\to (\neg q\to\neg p)$ is a tautology.
\end{enumerate}

\subsubsection{Classical Logic} Logical systems can be either intuitionistic or
classical, based on one fact: whether they accept the \emph{Law of Excluded
Middle} as an axiom. Precisely, it says

\newtheorem{axiom}{Axiom}[section]
\begin{axiom}[Law of Excluded Middle]
    For any proposition $p$, either $p$ or $\neg p$ holds.
\end{axiom}

Much of modern mathematics is built on classical logic, which asserts the axiom
of excluded middle. This leads to the existence of multitude of non-constructive 
proofs, which shows existence without giving a witness. A classical example of a
non-constructive proof using the law of excluded middle is given as follows:

\begin{theorem}
There exists irrational numbers $a, b$ where $a^b$ is rational.
\end{theorem}
\begin{proof}
By the law of excluded middle, $\sqrt{2}^{\sqrt{2}}$ is either rational or
irrational. 
    
Suppose it is rational, then take $a=b=\sqrt{2}$. We have found such $a,b$.

Suppose it is irrational, then take $a=\sqrt{2}^{\sqrt{2}}$, $b=\sqrt{2}$. Then
\[a^b={\left(\sqrt{2}^{\sqrt{2}}\right)}^{\sqrt{2}}=\sqrt{2}^{\sqrt{2}\times\sqrt{2}}=2\]
is rational.
\end{proof}

Even with the proof, we still don't have a concrete pair of irrational numbers
$(a,b)$ fulfilling the property, but the conclusion is declared to be logically
sound in classical logic. Intuitionistic logic, on the other hand, rejects this
rule, leading to a logical system where to prove something exists, it requires
an algorithm for the explicit construction of such object.

\subsection{The Correspondence}
The Curry-Howard correspondence is two-way: in particular, we can encode
a proposition as a type, and to write a proof is equivalent to 
finding a term that inhabit that particular type. This reduced the
problem of proof-checking to the problem of finding the type of a term,
or under the Curry-Howard correspondence, whether a proof truly proves
a given proposition.

This fundamental correspondence is elegant, but only limited to 
propositional logic. More complex type systems, such as the
Calculus of Constructions by Thierry Coquand, extend the Curry-Howard
correspondence to higher-order logic, finally allowing the encoding of
most mathematical truth and specifying proofs for theories in those 
systems.

\subsection{From STLC to PCUIC}
% \subsubsection{Simply Typed Lambda Calculus}

The Simply Typed Lambda Calculus (STLC) is a simple construction; we do not
define it here but refer to many classical textbooks such as
\sidecite{barendregt1992lambda} and \sidecite{girard1989proofs}. It is shown to
be equivalent to Intuitionistic Propositional Logic under the Curry-Howard
Correspondence. Furthermore, stronger type systems such as the Calculus of Constructions (CoC)
\sidecite{coquand1986calculus} extend the Curry-Howard correspondence to proofs 
in the full intuitionistic predicate logic (or \emph{1-st order logic}), allowing
proofs of quantified statements. The Polymorphic Cumulative Calculus of co-Inductive
Constructions (PCUIC) is a further extension of the CoC to include co-inductive
types, polymorphic universes and cumulativity, which we will not explain in this
paper and instead refer interested readers to the paper \sidecite{pcuic2017timany}.
