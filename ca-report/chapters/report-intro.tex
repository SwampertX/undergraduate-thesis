\chapter{Introduction}

The Coq Proof Assistant is one of the most prominent proof assistants, with
applications ranging from formalizing a mathematical theory with the Homotopy
Type Theory (HoTT)'s univalent foundation of mathematics
\sidecite{DBLP:journals/corr/BauerGLSSS16}, to a framework to verify a
computational theory via the Iris proof mode for Concurrent Separation Logic
\sidecite{jung2018iris}, to a practical large-scale verification project: the
recent recipient of the 2021 ACM Software System Award --- the CompCert
compiler \sidenote{\url{https://awards.acm.org/software-system}}. Although the
Coq Proof Assistant is the frameworking making these projects possible, one
might ask: "Who watches the watchers?" \sidenote{From latin: \emph{Quis
Custodiet ipsos custodes?}} Even though the theory behind Coq, the Polymorphic,
Cumulative Calculus of Inductive Constructions is known be consistent and
strongly normalizing (hence proof-checking is decidable), but the OCaml
implementation of Coq is known to have an average of one critical bug per year
which allows one to prove False statements in Coq.

The MetaCoq project \sidenote{\url{https://metacoq.github.io}} is therefore
started by the Gallinette Team in INRIA, to "formalize Coq in Coq" and acts as a
platform to interact with Coq's terms directly, in a verified manner; an
immediate application is to verify the correctness of the implementation of Coq.
This also reduces the trust in the implementation of Coq to the correctness of
the theories underlying Coq, moving from a "trusted code base" to a "trusted
theory base". In 2020, the core language of Coq, minus a few features are
already successfully verified in Coq \sidecite{coqcoqcorrect}. However, there
are still a few missing pieces not yet verified, among them the Module system of
Coq. The Module system, although not part of the core calculus of Coq, is an
important feature for Coq developers to develop in a modular fashion, providing
massive abstraction and a suitable interface for reusing definitions and
theorems.

In this project, I aim to extend the MetaCoq formalization of Coq by formalizing
the Module system of Coq in the MetaCoq project framework, by first
understanding the implementation of Modules, and then providing a specification
of the implementation of Coq modules, finally writing proofs to show the
correctness of the current implementation.  In particular, I will focus on the
formalization of non-parametrized, plain modules.

In order to tackle this project, it is important to understand the theory behind
the implementation of Coq; more explicitly, how Type Theory helps in theorem 
proving. I study under the supervision of Professor Yang Yue some results of
Type Theory; alongside the formalization of Modules jointly supervised by
Professor Nicolas Tabareau and Professor Martin Henz.

The first section of this report will give a brief mathematical overview of the
relationship between Type Theory and Theorem Proving. Following that, I
will give an introduction to the MetaCoq project, the language of Coq and
Coq Modules, before describing some related works and finally a
specification for Coq Modules which I will implement.