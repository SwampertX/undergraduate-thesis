\chapter{Project Plan for The Next Semester}
% \section{Project Plan for The Next Semester}

There are a few steps to this formalization of Coq Modules that I can foresee:
\begin{enumerate}
\item Study the structure and conventions of the MetaCoq project. Since this is 
    a huge project, many common theorems have been abstracted into multiple
    functors which can generate a theorem given a corresponding object, which
    I can make use of in my formalization.
\item Add the data structure of Plain Modules into the global environment which
    is quoted by TemplateCoq. This requires the modification of the current Global
    Environment data structure containing declarations (minus modules). 
\item Ensure the Global Environment remains well-defined with the current typing rules
    and correctness assertions. The modification in the previous step would require
    modifications to the existing 
    proofs on the Global Environment (well-formedness, well-typedness) and its 
    related properties.
\item Extend the TemplateCoq plugin (in OCaml) which transforms a Coq term into 
    a TemplateCoq representation, to translate a Coq module into the internal
    representation of Modules I implemented above. This requires OCaml knowledge
    specifically in the framework of Coq plugins.
\item Now that syntactical transformation from Coq to TemplateCoq is complete, I
    need to define how these modules that live in the Global Environment will 
    be used. In particular, I need to implement the "canonical kername" model as
    implemented in the Coq kernel to deal with module aliasing and usage of definitions.
\item The modification to the semantics in the previous step will again induce 
    changes to the theorems related to $\delta$-reductions (as explained in \ref{sec:using-modules}),
    so I have to extend the proofs to fit the new definitions. This should 
    ensure the TemplateCoq formalization of modules is sound.
\item Translate Coq Modules from TemplateCoq to PCUIC by elaboration. In PCUIC,
    modules will cease to exist and references to definitions in modules will be
    treated as direct references to plain definitions outside modules. This is
    done by referring module paths to their canonical paths.
\item The last step in this project is to verify that the translation from
    TemplateCoq to PCUIC is correct; that is, a PCUIC term is 
    well-formed and well-typed if and only if its translation is well-formed and well-typed
    in TemplateCoq.
\end{enumerate}

% \section{Study Type Theoretic Results}

% The modern field of Type Theory and Proof Assistants grows very quickly, with
% many new fields of logic (modal logic, linear logic, seperation logic) etc and
% their corresponding Type Theory being studied and made into proof assistants to 
% verify results in these fields. Since there were not many relevant courses in my 
% undergraduate curriculum, I decide to continue to study some classical results
% alongside with the formalization project above.

% The direction which I have chosen is towards a result important for the
% implementation of proof assistants: the decidability of conversion
% (correspondingly proof-checking) in a simpler type system, such as the Martin-Lof
% Type Theory. So far, I have covered results from Simply Typed Lambda Calculus
% including confluence and weak normalization, and its counterparts in proof theory,
% namely sequent calculus, natural deduction as a normal form, and cut-elimination.