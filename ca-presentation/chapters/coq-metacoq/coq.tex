\chapter{The Coq Proof Assistant}
% \section{The Coq Proof Assistant}

The Coq Proof Assistant (or Coq for short) is a proof assistant 
based on the Calculus of Inductive Constructions, which added
co-inductive types into the Calculus of Constructions.

\section{An example of a Coq program} This is a simple example featuring the
definition of a type ``nat'', a definition of a constant of type ``nat'', namely
``zero'', and a function ``plus''  which takes two ``nat''s and returns its sum.

Finally, thanks to Curry-Howard Correspondence, a proposition is a type; and
here a named proposition is postulated. A simple proof is also given, which
constructs a term inhabiting that type, hence proving the correctness of this
proposition.

\begin{listing}[!ht]
\inputminted{coq}{code/example.v}
\caption{A simple Coq program.}
\end{listing}

\section{Coq Modules}
Modules in Coq not only allows the reuse of code, it also provides parametrized
theories or data structures in the form of functors (or parametrized modules).

In the following example, we show how one can package definitions into named modules
for reuse; we also show how we can create parametrized generic data structures.
In the final line, a new Magma was created based on the Magma ``Nat''
by the functor ``DoubleMagma''.

\begin{listing}[!ht]
\inputminted{coq}{code/module_example.v}
\caption{An example of Modules.}
\end{listing}
