\section{Coq Modules}

\begin{frame}
    \frametitle{Example - Definitions}
    Modules as ``collections of definitions''.
    \inputminted[firstline=1,lastline=10]{coq}{code/module_example.v}
\end{frame}

\begin{frame}
    \frametitle{Example - Modules}
    ``Packaging'' definitions into a Module (Type).
    \inputminted[firstline=11,lastline=16]{coq}{code/module_example.v}
    \inputminted[firstline=19,lastline=23]{coq}{code/module_example.v}
\end{frame}

\begin{frame}
    \frametitle{Example - Aliasing}
    Modules can be aliased for ease of reference.
    \inputminted[firstline=24,lastline=25]{coq}{code/module_example.v}
\end{frame}

\begin{frame}
    \frametitle{Example - Functors}
    Higher-order modules - Functors.
    \inputminted[firstline=27]{coq}{code/module_example.v}
\end{frame}

\begin{frame}
    \frametitle{Towards a Specification of Plain Modules}
    \begin{itemize}
    \item There are many contention/issues regarding the semantics of functors:
    \underline{\href{https://github.com/coq/coq/wiki/OpenIssuesWithModules}{OpenIssuesWithModules}}
    \pause
    \item I only work on \textbf{plain} (non-parametrized) modules and module types, aliasing.
    \pause
    \item 2nd class object; cannot pass around as a term
    \item Restricted range of operations
    \begin{enumerate}
    \item Declaration/Definition (including aliasing)
    \item Using modules: access definitions in it using path names
    \item Refining a module: create new module by replacing entries of existing
    \end{enumerate}
    \end{itemize}
\end{frame}

\begin{frame}
    \frametitle{Representation of Global Environment and Modules}
    \begin{itemize}
    \item Before: global environment is a list of name-definition pairs for 
        constants and inductive types.\pause
    \item After: Modules are tree-like structures: they contain definitions of
    constants, inductives, modules, and module types. \pause
    \item Global environment is just a special case of a Module.
    \item Items in modules are referred to using pathnames (e.g. $M.a$)
    instead of defined names (e.g. $a$). The resolution of path name to its definition
    extends the $\delta$-reduction\footnote{Slightly different from named function
    application for lambda calculus with constants: plus 2 3$\to_\delta$ 5} in Coq.
    \item \textbf{Goal: Formalize the above in TemplateCoq.}
    \end{itemize}
\end{frame}

\begin{frame}
    \frametitle{Some Properties to Verify}
    \begin{enumerate}
    \item Definition: well-definedness of this structure: follow the Typing
    (Formation) Rules.
    \item Definition-Alias: well-definedness of aliasing: no cycles.
    \item Path: path points to the correct definition.
    \item Path-Alias: aliased path points to the correct definition.
    \item Related theorems about environment, typing, translation etc.

    \end{enumerate}
\end{frame}

\begin{frame}
    \frametitle{Example: TemplateToPCUICCorrectness.v}
    \inputminted{coq}{code/lemma.v}
\end{frame}
